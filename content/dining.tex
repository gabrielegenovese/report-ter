\paragraph{Dining Philosophers Example}
Let us consider a program with two participants playing the role of a
dining philosopher who shares two forks with the other participant.

\begin{lstlisting}
philosopher(Fork1, Fork2) ->
  send req to Fork1,
  receive ack from Fork1,
  send req to Fork2,
  receive ack from Fork2,
  eat(),
  send release to Fork1,
  send release to Fork2,
  philosopher(Fork1, Fork2).

fork() ->
  receive req from Phil,
  send ack to Phil,
  receive release from Phil,
  fork().
\end{lstlisting}
%
% Consider the following 
% %\footnote{The
% %corresponding Erlang code can be found in our
% %\href{https://github.com/gabrielegenovese/chorer/blob/example/deadlock/examples/paper_example/deadlock/deadlock.erl}{github}
% %repository.} 
% program, where each function defines the behavior of a
% participant in the communicating system, which can be instantiated
% multiple times.  The program models two dining philosophers.
%

%
The behavior of the philosophers is given by the pseudocode on the
right while pseudocode for the behavior of the forks is discussed below.
Each philosopher first acquires the forks (starting with the one on
its right, that is the one with the same index), then eats (with the
function call \lstinline{eat()} representing some terminating local
computation), and finally releases the forks before recurring.
Parameters \lstinline{Fork1} and \lstinline{Fork2} are references to
processes executing the behavior of forks described by the pseudocode on the left
that repeatedly waits for the request from process \lstinline{Phil},
\lstinline{ack}s the request, and waits for the \lstinline{release}
message from \lstinline{Phil}.


There are two possible behaviors of the system. The first (good) one where the philosophers alternate eating infinitely and a second (bad) behavior 
%three
%possible executions of the system. In the first one, the first
%philosopher takes both forks, eats, then releases them,
%allowing the second philosopher to do the same. In the second
%execution, the philosophers eat in the opposite order. Given the recursive behavior of the \texttt{philosopher} 
%function, the actors could alternate infinitely.
%In the third possible execution, 
where both philosophers
manage to take only one fork each %, but they cannot take the other, as they
%are waiting for the other philosopher to release it. This 
resulting in a deadlock. 
Figure~\ref{graph:philosophers} depicts the global Choreography Automaton
representing the program above. 
We have two recursive executions where both philosophers
eat, represented as loops, and two executions which end in the same
deadlock state.
%\footnote{The full global view,
%obtained with our prototype tool Chorer \cite{chorer}, can be seen in
%our
%\href{https://github.com/gabrielegenovese/chorer/blob/example/deadlock/examples/paper_example/deadlock/main_0_global_view.dot}
  %   {github} repository.}.
%
%In the automaton, State 1 is the entry point. From State 1, we have
%two possible transitions, to State 2 and State 3 respectively, which
%model the requests (\texttt{req}) of the philosophers to the forks
%with the corresponding index. The two transitions can occur in any
%order (indeed there are transitions to State 5 closing the commuting
%diamond).
%
%Transitioning from State 1 to State 2 means that the system first 
%performs the \texttt{req} communication for the first philosopher. 
%Transitioning from State 1 to State 3 means that it first performs 
%the \texttt{req} communication for the second philosopher.
%States 2 and 3 represent states where there is a race between the
%philosopher that already has one fork taking the other (transitions on
%the side, leading to a loop), or the other philosopher taking the
%fork, leading to the deadlock in State 5. 
The deadlock is 
visible since State 5 is not final, and it has no outgoing transitions.
%Also, the system has a race, this is visible in the choreography
%automaton since State 3 has two outgoing transitions with the same
%receiver (the same for State 2), but we want to see these transitions
%since some of them lead to the deadlock state. 
  The automaton is simplified %w.r.t.~a full model of the program, 
  as it does not display the \texttt{ack} messages, and the
  \texttt{release} messages are merged since their order is
  irrelevant. The focus here is solely on the order of the
  \texttt{req} messages. One can imagine to first extract the full
  choreography automaton, and then merge equivalent behaviors and abstract away from uninteresting transitions.
%  to focus on the ones of interest. 
%  Another possible simplification
%  would be to notice that the system is symmetric (w.r.t.~the swap of
%  the two philosophers), hence it would be enough to analyze one of
%  the two halves.



\begin{figure}[t]
  \centering
  % \includegraphics[scale=.35]{images/deadlock.png}
  % \makebox[\textwidth][c]{
  \resizebox{0.9\textwidth}{!}{%
      \begin{tikzpicture}[node distance={40mm}, thick, main/.style = {draw, circle}] 
        \node[state] (n_1) {1};
        \node[state] (n_2) [below right of=n_1] {2};
        \node[state] (n_3) [below left of=n_1] {3};
        \node[state] (n_4) [right of=n_2] {4};
        \node[state] (n_5) [below of=n_1, fill=red] {5};
        \node[state] (n_6) [left of=n_3] {6};
        
        \draw[->] (n_1) -- node[midway, right, pos=0.5] {philo0→fork0:req} (n_2);
        \draw[->] (n_1) -- node[midway, left, pos=0.5] {philo1→fork1:req} (n_3);
        \draw[->] (n_2) -- node[midway, below right, pos=0.5] {philo1→fork1:req} (n_5);
        \draw[->] (n_3) -- node[midway, below left, pos=0.5] {philo0→fork0:req} (n_5);
        \draw[->] (n_2) -- node[midway, above, pos=0.5] {philo0→fork1:req} (n_4);
        \draw[->] (n_4) to[bend right=20] node[midway, right, pos=0.5] {philo0→fork\{0,1\}:release} (n_1);
        \draw[->] (n_3) -- node[midway, above, pos=0.5] {philo1→fork0:req} (n_6);
        \draw[->] (n_6) to[bend left=20] node[midway, left, pos=0.5] {philo1→fork\{0,1\}:release} (n_1);
      \end{tikzpicture}
    }
  % }%
  \caption{Global view of the dining philosophers example.}
  \label{graph:philosophers}
\end{figure}