\paragraph{Dining Philosophers Example}
This example is taken from the \texttt{dining} case study from the benchmark 
suite of the tool.
Let us consider a program with two participants playing the role of a
dining philosopher who shares two forks with the other participant.

\begin{lstlisting}
philosopher(Fork1, Fork2) ->
  send req to Fork1,
  receive ack from Fork1,
  send req to Fork2,
  receive ack from Fork2,
  eat(),
  send release to Fork1,
  send release to Fork2,
  philosopher(Fork1, Fork2).

fork() ->
  receive req from Phil,
  send ack to Phil,
  receive release from Phil,
  fork().
\end{lstlisting}

The behavior of the philosophers is given by the pseudocode on the
right while pseudocode for the behavior of the forks is discussed below.
Each philosopher first acquires the forks (starting with the one on
its right, that is the one with the same index), then eats (with the
function call \lstinline{eat()} representing some terminating local
computation), and finally releases the forks before recurring.
Parameters \lstinline{Fork1} and \lstinline{Fork2} are references to
processes executing the behavior of forks described by the pseudocode on the left
that repeatedly waits for the request from process \lstinline{Phil},
\lstinline{ack}s the request, and waits for the \lstinline{release}
message from \lstinline{Phil}.

There are two possible behaviors of the system. The first (good) one where the philosophers alternate eating infinitely and a second (bad) behavior 
where both philosophers
manage to take only one fork each
resulting in a deadlock. 
Figure~\ref{graph:philosophers} depicts the global Choreography Automaton
representing the program above. 
We have two recursive executions where both philosophers
eat, represented as loops, and two executions which end in the same
deadlock state.
The deadlock is 
visible since State 5 is not final, and it has no outgoing transitions.
The automaton is simplified 
as it does not display the \texttt{ack} messages, and the
\texttt{release} messages are merged since their order is
irrelevant. The focus here is solely on the order of the
\texttt{req} messages. One can imagine to first extract the full
choreography automaton, and then merge equivalent behaviors and abstract 
away from uninteresting transitions.

\begin{figure}[t]
  \centering
  % \includegraphics[scale=.35]{images/deadlock.png}
  % \makebox[\textwidth][c]{
  \resizebox{0.9\textwidth}{!}{%
      \begin{tikzpicture}[node distance={40mm}, thick, main/.style = {draw, circle}] 
        \node[state] (n_1) {1};
        \node[state] (n_2) [below right of=n_1] {2};
        \node[state] (n_3) [below left of=n_1] {3};
        \node[state] (n_4) [right of=n_2] {4};
        \node[state] (n_5) [below of=n_1, fill=red] {5};
        \node[state] (n_6) [left of=n_3] {6};
        
        \draw[->] (n_1) -- node[midway, right, pos=0.5] {philo0→fork0:req} (n_2);
        \draw[->] (n_1) -- node[midway, left, pos=0.5] {philo1→fork1:req} (n_3);
        \draw[->] (n_2) -- node[midway, below right, pos=0.5] {philo1→fork1:req} (n_5);
        \draw[->] (n_3) -- node[midway, below left, pos=0.5] {philo0→fork0:req} (n_5);
        \draw[->] (n_2) -- node[midway, above, pos=0.5] {philo0→fork1:req} (n_4);
        \draw[->] (n_4) to[bend right=20] node[midway, right, pos=0.5] {philo0→fork\{0,1\}:release} (n_1);
        \draw[->] (n_3) -- node[midway, above, pos=0.5] {philo1→fork0:req} (n_6);
        \draw[->] (n_6) to[bend left=20] node[midway, left, pos=0.5] {philo1→fork\{0,1\}:release} (n_1);
      \end{tikzpicture}
    }
  % }%
  \caption{Global view of the dining philosophers example.}
  \label{graph:philosophers}
\end{figure}