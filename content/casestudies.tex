In order to better illustrate our point, we present in this section two examples (using Erlang-like, actor-based pseudocode), each one
composed by a pseudocode and the corresponding Choreography
Automaton~\cite{coordination2020-chorAuto} (a graphical way of
representing a finite-state machine) of the global view of the
communicating system. Here, a \emph{global view} is an abstract description of
all the possible behaviors of the full system. We consider a language
where receive statements are blocking operations. A state where each
participant has completed its task and terminated is called \emph{final}. As such, a
state that is not final and has no outgoing transitions is a \emph{deadlock}.

The first example is a concise reproduction of the dining philosophers
problem, which highlights a possible deadlock. The second example
shows a possible mutual exclusion error 
when operating a simple bank account.
%in a database.
%
Both examples are available online~\cite{chorer_examples}; the
corresponding automata have been obtained with the help of our prototype tool
Chorer~\cite{chorer}.

\lstset{language=erlang, basicstyle=\sffamily\footnotesize,
  keywordstyle=\color{blue}, numberstyle=\tiny, numbers=none,
  showspaces=false, showstringspaces=false, frame=tL,
  backgroundcolor=\color{black!5}, morekeywords={send, to, from} }


In the examples, we exploit two operations for sending and receiving messages,
respectively.
%
More precisely, \lstinline[language=erlang, morekeywords={send,
  to}]{send msg to proc} sends message \lstinline{msg} to process
\lstinline{proc}, while \lstinline[morekeywords={receive,
  from}]{receive pat1 from proc1 -> e1;...;patn from procn -> en} 
  represents a branching point where the process receives 
  the first message 
  that matches a pattern $\mathsf{pati}$ and, then, 
  continues with the execution of $\mathsf{ei}$. 
  As in Erlang, 
  %we allow \lstinline[language=erlang]{receive}
%statement with multiple clauses where the 
pattern matching is tried from top to bottom.
% when there are
%multiple clauses.
When a receive has only one clause, we abbreviate it as
\lstinline[morekeywords={receive,from}]{receive pat from proc}
(and continues with the execution of the next sentence).