\textbf{Choreographies} are a formal model used to 
represent systems of communicating processes, enabling semantic proofs 
regarding the presence or absence of the mentioned properties. Choreographies 
are \textbf{global views} of the behavior of a system, giving a comprehensive 
perspective of the communication exchanges among actors (also called 
participants). From the global view, via a simple projection, one can obtain 
the \textbf{local view}, i.e., the individual behavior of each participant in 
the communication process. Notice that, the local view is limited and actors 
are unaware of the behavior of the rest of the system.

A \textit{visual} way to formalize choreographies is through 
\textbf{Choreographic Automata} \cite{coordination2020-chorAuto}, which use 
finite-state automata to describe communication systems. This representation 
effectively illustrates program flow, showing loops and branching, while 
leveraging existing results \cite{orlando2021corinne}.

This project aims to enhance a tool that extracts choreographic specifications 
from Erlang programs as Choreographic Automata. The process involves deriving 
local views for each actor and composing them into a global choreography, an 
inverse operation w.r.t. projection, which may not always be feasible. The 
following example illustrates this extraction. This example differs from the
one shown in the previous Description of Work, and it's taken from the benchmark
suite.

\begin{exmp}[Async example]
In this example, two actors, \texttt{dummy1} and \texttt{dummy2}, exchange a 
message asynchronously, meaning that we don't know which message will arrive 
first. Because the \textit{send} operation is non-blocking, and we don't know 
which \textit{receive} will be performed first, the expected graph should show 
two possible lines of execution: one where the exchange of \texttt{ciao} occurs 
first and one where the exchange of \texttt{bello} occurs first.

\bigskip

\begin{lstlisting}[language=Erlang, caption=Two processes exchanging messages 
asynchronously, label=code:async]
-module(async).
-export([main/0, dummy1/0, dummy2/0]).

dummy1() ->
    d2 ! bello,
    receive
        ciao -> done
    end.

dummy2() ->
    d1 ! ciao,
    receive
        bello -> done
    end.

main() ->
    A = spawn(?MODULE, dummy1, []),
    register(d1, A),
    B = spawn(?MODULE, dummy2, []),
    register(d2, B).
\end{lstlisting}
\todo{codice da spiegare: meglio usare uno pseudocodice?}

Listing \ref{code:async} shows to the described scenario in Erlang. Figures 
\ref{local:main}, \ref{local:dummy1a} and \ref{local:dummy2a} depict the local
views of the actors.

\begin{figure}[!ht]
    \centering
    \resizebox{.8\textwidth}{!}{%
        \begin{tikzpicture}[node distance={33mm}, thick, main/.style = {draw,circle}] 
          \node[initial, state] (n_1) {1};
          \node[state] (n_2) [right of=n_1] {2};
          \node[state,accepting] (n_3) [right of=n_2] {3};
          
          \draw[->] (n_1) -- node[midway, above, pos=0.5] {$\Delta$ dummy1} (n_2);
          \draw[->] (n_2) -- node[midway, above, pos=0.5] {$\Delta$ dummy2} (n_3);
        \end{tikzpicture}
    }
    \caption{Local view of the \texttt{main} actor}
    \label{local:main}
\end{figure}

\begin{figure}[!ht]
    \centering
    \resizebox{.8\textwidth}{!}{%
        \begin{tikzpicture}[node distance={40mm}, thick, main/.style = {draw,circle}] 
          \node[initial, state] (n_1) {1};
          \node[state] (n_2) [right of=n_1] {2};
          \node[state, accepting] (n_3) [right of=n_2] {3};
          
          \draw[->] (n_1) -- node[midway, above, pos=0.5] {d2 ! bello} (n_2);
          \draw[->] (n_2) -- node[midway, above, pos=0.5] {receive ciao} (n_3);
        \end{tikzpicture}
    }
    \caption{Local view of the \texttt{dummy1} actor}
    \label{local:dummy1a}
\end{figure}

\begin{figure}[!ht]
    \centering
    \resizebox{.8\textwidth}{!}{%
        \begin{tikzpicture}[node distance={40mm}, thick, main/.style = {draw,circle}] 
          \node[initial, state] (n_1) {1};
          \node[state] (n_2) [right of=n_1] {2};
          \node[state, accepting] (n_3) [right of=n_2] {3};
          
          \draw[->] (n_1) -- node[midway, above, pos=0.5] {d1 ! ciao} (n_2);
          \draw[->] (n_2) -- node[midway, above, pos=0.5] {receive bello} (n_3);
        \end{tikzpicture}
    }
    \caption{Local view of the \texttt{dummy2} actor}
    \label{local:dummy2a}
\end{figure}

After associating the local views with the actors, the algorithm generates 
the global view shown in Figure \ref{global:async}. 
The automaton expresses two possible global 
executions of the program: from State 1 to State 3, the actors \texttt{dummy1} 
and \texttt{dummy2} are spawned. Then, since the \textit{send} operation is 
non-blocking, either \texttt{dummy1} sends \texttt{ciao} first (leading to 
State 6) or \texttt{dummy2} sends \texttt{bello} first (leading to State 4). 
Consequently, the final state is reached through two possible paths, 
depending on the order in which messages are received.

\begin{figure}[!ht]
    \centering
    \resizebox{\textwidth}{!}{%
        \begin{tikzpicture}[node distance={30mm},thick,main/.style={draw,circle}] 
          \node[initial, state] (n_1) {1};
          \node[state] (n_2) [right of=n_1] {2};
          \node[state] (n_3) [right of=n_2] {3};
          \node[state] (n_4) [above right of=n_3] {4};
          \node[state,accepting] (n_5) [right of=n_4] {5};
          \node[state] (n_6) [below right of=n_3] {6};
          \node[state,accepting] (n_7) [right of=n_6] {7};
          
          \draw[->] (n_1) -- node[midway, above, pos=0.5] {main$\Delta$d1} (n_2);
          \draw[->] (n_2) -- node[midway, above, pos=0.5] {main$\Delta$d2} (n_3);
          \draw[->] (n_3) -- node[midway, left, pos=0.5] {d2$\to$d1:bello} (n_4);
          \draw[->] (n_4) -- node[midway, above, pos=0.5] {d1$\to$d2:ciao} (n_5);
          \draw[->] (n_3) -- node[midway, left, pos=0.5] {d1$\to$d2:ciao} (n_6);
          \draw[->] (n_6) -- node[midway, above, pos=0.5] {d2$\to$d1:bello} (n_7);
        \end{tikzpicture}
    }
    \caption{Global view of Listing \ref{code:async}}
    \label{global:async}
\end{figure}
\end{exmp}

% inserire in appendice le definizioni, la grammatica, algoritmi
% link all'appendice per le versioni formali
% esempi che fanno vedere le differenze tra il prima e il dopo