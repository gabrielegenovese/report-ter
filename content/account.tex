This case study is taken from the \texttt{account} example from the benchmark 
suite of the tool.
We now consider a system where a bank account is accessed
by two clients, dubbed C1 and C2.
% 
\tikzset{
	hpath/.style={
			very thick,
			line cap = round,
			line join = round,
			line width=0.1cm,
			opacity=.70,
			color = teal!30
		}
}

\bigskip

\begin{lstlisting}
account(Value) ->
  receive
    read from Client ->
      send Value to Client,
      account(Value);
    NewValue from Client ->
      account(NewValue).

client() ->
  send read to Acc,
  receive Value from Acc,
  % operations on Value
  send NewValue to Acc.
\end{lstlisting}

\bigskip

The pseudocode yields the behavior of the bank account,
where $\mathsf{Value}$ represents its current balance.
This process waits for requests from a client.
A request can either be a \lstinline{read} access to know
the current balance or an update request of such value to a
\lstinline{NewValue}.

Symmetrically, client processes C1 and C2 behave according to the
pseudocode on the left: the process reads the current balance
from the account,
performs some internal operations based on such value, and
updates the balance.
%
The global view of the communicating system is depicted in
Figure~\ref{fig:account}.

\newcommand\dummy{C}
\begin{figure}[!ht]
	\centering
	\begin{tikzpicture}[node distance={27mm}, scale = .6, transform shape, thick, main/.style = {draw, circle}]
		\node (n_1) [state] {};
		\node (n_2) [state, below left of=n_1] {};
		\node (n_3) [state, below right of=n_1] {};
		\node (n_4) [state, left= 3.5cm of n_2] {4};
		\node (n_5) [state, below=3cm of n_1] {};
		\node (n_6) [state, right= 3.5cm of n_3] {6};
		\node (n_7) [state, below of=n_4] {7};
		\node (n_8) [state, below left of=n_5] {};
		\node (n_9) [state, below right of=n_5] {};
		\node (n_10) [state, below of=n_6] {10};
		\node (n_11) [state, accepting, below= of n_7] {11};
		\node (n_12) [state, accepting, below=1.5cm of n_8] {};
		\node (n_13) [state, accepting, below= 1.5cm of n_9] {};
		\node (n_14) [state, accepting, below=2.5cm of n_10] {14};

		\draw[->] (n_1) -- node[midway, above left] {acc→\dummy1:Value} (n_2);
		\draw[->] (n_2) -- node[midway, below left] {acc→\dummy2:Value} (n_5);
		\draw[->] (n_5) -- node[midway, above left=-2mm] {\dummy1→acc:NewValue} (n_8);
		\draw[->] (n_8) -- node[midway, below left] {\dummy2→acc:NewValue} (n_12);

		\draw[->] (n_1) -- node[midway, above right] {acc→\dummy2:Value} (n_3);
		\draw[->] (n_3) -- node[midway, below right] {acc→\dummy1:Value} (n_5);
		\draw[->] (n_5) -- node[midway, above right=-2mm] {\dummy2→acc:NewValue} (n_9);
		\draw[->] (n_9) -- node[midway, below right] {\dummy1→acc:NewValue} (n_13);

		\draw[hpath] ($(n_1.center) + (-5pt,-5pt)$) -- (n_2.center) -- (n_5.north west) -- ($(n_5.center)+(-5pt,0)$) -- (n_5.south west) -- (n_8.center) -- ($(n_12.center) + (0pt,5pt)$);
		\draw[hpath] ($(n_1.center) + (5pt,-5pt)$) -- (n_3.center) -- (n_5.north east)  -- ($(n_5.center)+(5pt,0)$) -- (n_5.south east)  -- (n_9.center) -- ($(n_13.center) + (0pt,5pt)$);

		\foreach \n in {1,2,3,5,8,9,12,13}{
				\node at (n_\n) {\n};
			}

		\draw[->] (n_2) -- node[midway, above=3mm] {C1→acc:NewValue} (n_4);
		\draw[->] (n_4) -- node[midway, above left] {acc→C2:Value} (n_7);
		\draw[->] (n_7) -- node[midway, above left] {C2→acc:NewValue} (n_11);

		\draw[->] (n_3) -- node[midway, above=3mm] {C2→acc:NewValue} (n_6);
		\draw[->] (n_6) -- node[midway, above right] {acc→C1:Value} (n_10);
		\draw[->] (n_10) -- node[midway, above right] {C1→acc:NewValue} (n_14);
	\end{tikzpicture}
	\caption{Global view of the bank account example}
	\label{fig:account}
\end{figure}

We can observe two correct executions where the operations are
performed in a read-update-read-update order (taking the path via
states 1-2-4-7-11 or the one via states 1-3-6-10-14).
%

However, there is also a read-read-update-update order on the
highlighted paths.
%
Although the choreography is not inherently incorrect, these
highlighted paths could represent a violation of mutual exclusion
which may be undesirable for developers in certain
contexts.
The choreography automaton in Figure~\ref{fig:account} helps in spotting
this issue.

%%% Local Variables:
%%% mode: LaTeX
%%% TeX-master: "main"
%%% End:
