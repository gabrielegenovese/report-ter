To better understand how the parsing of a file and the creation of a localview 
should work, we created an attributed grammar for a subset of the Erlang 
language. An attributed grammar extends a context-free grammar by associating 
attributes with its symbols and defining semantic rules. Attributes can be 
\textit{synthesized} (computed from child nodes) or \textit{inherited}
(passed from parent nodes). 
The attribute part can be seen as a set of operations that the compiler  
performs during its analysis. This serves two key purposes: it provides  
a clear understanding of what the tool can parse from the programming  
language, and it defines how information propagates through the parse  
tree. This propagation helps identify the core data structures that  
underpin the tool’s functionality.  

The following attributed grammar will be presented alongside comments  
to provide insights into the tool’s behavior. First, we introduce the  
grammar fragment, followed by the corresponding operations applied  
to the local view data structure. These operations illustrate how the  
tool processes and interprets parsed elements, highlighting its  
internal mechanisms. 
The \texttt{link} function is used to establish a logical connection  
between two nodes. Its third argument defines the label of the transition.  
If this argument is not provided, the transition is considered an  
$\epsilon$-transition, representing an unlabeled connection. 

\paragraph{Program} For simplicity, we consider an Erlang program as a set 
of one or more functions, that can be called by the user. All the needed 
function are within the program.

\bigskip

\noindent $prog \to (function)^+$

A complete program does not have a local view, so there is no need  
to define attributes.  

\paragraph{Function}  
A function can have its own local view representation.  
In Erlang, pattern matching allows multiple definitions,  
so we must distinguish between the base case and the  
pattern matching case. The base case is straightforward,  
so we only present the pattern matching case.  

\bigskip

\noindent Base case: $function \to fun.$

\noindent With multiple definitions: $function \to fun_1;...;fun_N.$

\begin{verbatim}
function.nodes = new U fun1.nodes U ... U funN.nodes
function.edges = link(new,fun1.first) U ... U 
                 link(new,funN.first) U 
                 fun1.edges U ... U funN.edges
function.first = new
\end{verbatim}

Each definition has its own local view representation, so we must link  
each one to a new state. This state serves as the starting point of  
the entire local view. The nodes consist of all other nodes, and  
the same applies to the edges.  

\paragraph{Function body}  
The body of a function consists of a name (which is an atom),  
a set of arguments (which may be empty), and a list of expressions.  

\bigskip

\noindent $fun \to Atom(X_1,...,X_n) \to Exprs$

\begin{verbatim}
fun.nodes = Exprs.nodes
fun.edges = Exprs.edges
fun.first = Exprs.first
fun.last = Exprs.last
fun.context = [ X1 -> Param[1], ..., Xn -> Param[n] ]
fun.ret_var = Exprs.ret_var
\end{verbatim}

When encountering a function, we simply add its arguments to the context.  
The \texttt{Param} is taken as input, with a default value of \texttt{ANYDATA}.  
Additionally, we can perform semantic checks to verify the function’s  
existence.  

\paragraph{Expressions}  
A list of expressions can have one or more elements. The single-expression  
case is straightforward, so its attributes are omitted.  

\bigskip

\noindent Base case: $Exprs \to expr$

\noindent With multiple expression: $Exprs \to expr,Exprs'$

\begin{verbatim}
Exprs.nodes = expr.nodes U Exprs'.nodes
Exprs.edges = expr.edges 
              U Exprs'.edges 
              U link(expr.last, Exprs'.first)
Exprs.first = expr.first
Exprs.last = Exprs'.last
Exprs.context = expr.context U Exprs'.context
Exprs.ret_var = Exprs'.ret_var
\end{verbatim}

Expressions represent the lines of code in an Erlang program.  
We cover only the essential ones (e.g., assignments, function calls)  
and those related to communication.  

\paragraph{Send}   
As mentioned earlier, a send should add a transition to the local graph.  

\bigskip

\noindent $expr \to expr'\ \ !\ \ expr''$

\begin{verbatim}
expr.nodes = expr'.nodes U expr''.nodes U new1 U new2
expr.edges = expr'.edges U expr''.edges
             U link(expr''.last, new1)
             U link(new1, new2, expr'.ret_var 
                  + " ! " 
                  + expr''.ret_var)
expr.first = expr'.first
expr.last = new2
expr.context = expr'.context U expr''.context
expr.ret_var = expr''.ret_var
\end{verbatim}

In Erlang, both the left-hand side and right-hand side of an operation  
can be expressions. First, the left-hand side expression is evaluated,  
followed by the right-hand side. Finally, a send edge is inserted with  
the corresponding variable as the label. The left-hand side must contain  
a process identifier.  

\paragraph{Receive}  
Like for a send, a receive operation should add a transition to  
the local graph for each matching pattern.  

\bigskip

\noindent $expr \to receive patter_1 \to Exprs_1; ...; pattern_n \to Exprs_n end$

\begin{verbatim}
expr.first = new1
expr.last = new2
expr.nodes = Exprs1.nodes U ... U Exprsn.nodes U new1 U new2
expr.edges = Exprs1.edges U ... U Exprsn.edges
             U link(new1, Exprs1.first) 
             U ...
             U link(new1, Exprsn.first)
             U link(Exprs1.last, new2)
             U ...
             U link(Exprsn.last, new2, espilon)
\end{verbatim}

A \texttt{receive} operation can contain an entire block of code 
that must be evaluated.  
Its return data cannot be determined statically, as it varies depending  
on the matching pattern during the global view phase. Additionally,  
since the context is modified during global view simulation,  
we leave these two attributes unchanged.  

\paragraph{Spawn call}  
The same reasoning used before applies to the \texttt{spawn} function.  

\bigskip

\noindent $expr \to spawn(Atom, Params)$

\begin{verbatim}
expr.nodes = new1 U new2
expr.edges = link(new1, new2, "spawn Atom")
expr.first = new1
expr.last = new2
expr.ret_var = newVar(type: pid, value: random) 
\end{verbatim}

The \texttt{spawn} function returns a new variable as output.  
Its type should be \texttt{pid}, with no assigned name.  
An additional logical identifier is included to differentiate  
it during global view construction. The context is left untouched.

\paragraph{Recursive call}  
Recursive calls must be considered, as they are fundamental  
in functional languages like Erlang.  

\bigskip

\noindent $expr \to FunName(Params)$

\begin{verbatim}
expr.edges = expr.edges U link(FunName.first, expr.last)
expr.ret_var = null
\end{verbatim}

For recursive calls, the last added node must be linked to the first  
node of the local view being created. Using the function name,  
we can retrieve its previously created first node.  
Since a single analysis cannot determine the function's return type,  
we set the return variable to \texttt{null}. After evaluating the  
recursive call, the analysis stops, allowing only tail-recursive  
functions.  

\paragraph{Generic function call}  
To complete the previous analysis, we define attributes for  
standard function calls.  

\bigskip

\noindent $expr \to Atom(Params)$

\begin{verbatim}
G = get_localview(Atom, Param)
expr.nodes = G.nodes
expr.edges = G.edges
expr.first = G.first
expr.last = G.last
expr.ret_var = G.ret_var 
\end{verbatim}

With \texttt{get\_localview}, we retrieve the necessary information  
to complete the local view for a generic function call.  

\paragraph{Assignment}  
Finally, assignments must be handled to support basic programs  
with variables. Assignments in Erlang are peculiar as it has
immutable variables: they behave  
as pattern matching when a variable is already defined.
Thus, semantic checks can be performed.  

\bigskip

\noindent $expr \to pattern = expr'$

\begin{verbatim}
expr.nodes = expr'.nodes
expr.edges = expr'.edges
expr.first = expr'.first
expr.last = expr'.last
expr.context = [pattern -> expr'.ret_var]
\end{verbatim}

The pattern should add the new variable to the context  
with its assigned name.  

\paragraph{Conclusion}
%% TODO
