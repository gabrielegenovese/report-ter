\paragraph{How to use the tool}
The tool can be used from the command line interface (CLI). It can be used by
compiling each module from the Erlang Shell (Eshell) or, more conveniently, by
using \texttt{rebar3}, a standard build tool that provides various features
such as package management for community-created libraries, compilation, and
automated project testing. By cloning the project from the public GitHub
repository and running the \texttt{rebar3 escriptize} command in the main
directory, \texttt{rebar3} will, in this order, check for and download
dependencies if needed, compile the project, and run tests if present. After
that, an executable can be used in \texttt{./\_build/default/bin/chorer}.

\begin{lstlisting}[caption=Usage message]
Usage:
  chorer <input> <entrypoint> <output> <ming> <gstate> <minl>

Extract a choreography automata of an Erlang program.

Arguments:
  input      Erlang source file (string)
  entrypoint Entrypoint of the program (atom)
  output     Output directory for the generated dot files (string), default: ./
  ming       Minimize the globalviews , default: false
  gstate     Global state are formed with previous messages , default: true
  minl       Minimize the localviews , default: true
\end{lstlisting}

\noindent The mandatory arguments are:
\begin{itemize}
    \item \textbf{Input}: The relative path string of the input Erlang program,
    from which the tool will generate local and global views.
    \item \textbf{Entrypoint}: The atom representing the function where the
    execution of the input program begins. This parameter is essential because
    Erlang does not have a conventional entry point function (i.e. the main in C,
    but in Erlang can be every exported function). It will be passed to the
    function that creates the global view to start the simulation.
\end{itemize}

\noindent The optional arguments are:
\begin{itemize}
    \item \textbf{Output}: The relative path string of the output folder where
    the local and global view files will be saved. Local views will be named
    \texttt{[function name with arity]\_local\_view.dot}, while the global view
    file will be named \texttt{[Entrypoint]\_global\_view.dot}.
    \item \textbf{Options}: Currently, this is a tuple of two booleans that
    allow customization of local views. The first boolean determines whether to
    identify final states (by default, it is set to true). The second adds more
    information to the local view by including transitions related to other
    language constructs in addition to communication constructs (by default, it
    is set to false). This parameter may be expanded in the future with
    additional booleans.
\end{itemize}

\begin{lstlisting}[language=Erlang, caption=Use example of the tool]
shell> ./_build/default/bin/chorer ./path/to/file.erl entrypoint/0
...
[INFO] Finished!
\end{lstlisting}

\paragraph{Struttura del tool}
The project is divided into two main folders. Under the folder \texttt{examples},
there are various examples of Erlang programs to test the tool. The tool's code
is located in the \texttt{src} folder.

\bigskip

\dirtree{%
.1 chorer.
.2 examples.
.3 ....
.2 src.
.3 choreography.
.4 actor\_emul.erl.
.4 eval.erl.
.4 gv.erl.
.4 lv.erl.
.4 md.erl.
.3 common.
.4 common\_data.hrl.
.4 db.erl.
.4 digraph\_to\_dot.erl.
.4 fsa.erl.
.4 log.erl.
.4 settings.erl.
.4 share.erl.
.3 chorer\_app.erl.
.3 chorer.app.src.
.2 rebar.config.
.2 rebar.lock.
}

\bigskip

The main file is \texttt{chorer\_app.erl}, which contains the \texttt{main}
function. The file \texttt{rebar.config}, \texttt{rebar.lock}, and
\texttt{chorer.app.src} are required for running the tool with \texttt{rebar3}.
The \texttt{choreography} folder contains modules responsible for the static
analysis of the program and the creation of local and global views. Specifically:

\begin{itemize}  
    \item \texttt{actor\_emul.erl}: manages the emulation of an actor during the
    globalview creation.
    \item \texttt{eval.erl}: helps the localview generation.  
    \item \texttt{gv.erl}: generates the globalview.  
    \item \texttt{lv.erl}: generates the localviews.  
    \item \texttt{md.erl}: manages the extraction of preliminary data.  
\end{itemize}  

\noindent The \texttt{share} folder contains shared functions for all modules,
including:  
\begin{itemize}  
  \item \texttt{common\_data.erl}: header file where the main data structures
  are defined.  
  \item \texttt{db.erl}: manages the common information used during the creation
  of both local and global view.  
  \item \texttt{digraph\_to\_dot.erl}: converts a graph into DOT format.  
  \item \texttt{fsa.erl}: manages and minimizes automata.  
  \item \texttt{log.erl}: manages the logging of messages.  
  \item \texttt{settings.erl}: manages the settings of the program.  
  \item \texttt{share.erl}: handles file saving and shared function.  
\end{itemize}

\subsubsection{From Erlang to Choreographies}
\label{sec:corrisp}

\paragraph{Localview}

The code of a \texttt{receive} operation corresponds to the Figure \ref{grafo:receive}:
each branch will be evaluated recursively, continuing the construction from the 
branches. Finally, all branches will merge into a common node with $\epsilon$ 
transitions, from which the evaluation of the local view will resume.  
The send operation (\texttt{Pid ! message}) operation corresponds to Figure \ref{grafo:send}.  
The \texttt{spawn} function call corresponds to Figure \ref{grafo:spawn}.  

\begin{figure}[ht!]
    \centering
    \begin{tikzpicture}[node distance={29mm}, thick, main/.style = {draw, circle}] 
      \node[state] (n_1) {1};
      \node[state] (n_2) [above right of=n_1] {2};
      \node[state] (n_4) [below right of=n_1] {$n$};
      \node[state] (n_3) [below=1.1\distance cm of n_2] {\ldots};
      \node[state] (n_5) [right of=n_2] {\ldots};
      \node[state] (n_6) [right of=n_3] {\ldots};
      \node[state] (n_7) [right of=n_4] {$m$};
      \node[state] (n_8) [below right of=n_5] {$m+1$};
      
      \draw[->] (n_1) -- node[midway, above left, pos=0.6] {0\#receive message1} (n_2);
      \draw[->] (n_1) -- node[midway, above, pos=0.5] {...} (n_3);
      \draw[->] (n_1) -- node[midway, below left, pos=0.6] {N\#receive messageN} (n_4);
      \draw[->] (n_2) -- node[midway, above, pos=0.5] {...} (n_5);
      \draw[->] (n_3) -- node[midway, above, pos=0.5] {...} (n_6);
      \draw[->] (n_4) -- node[midway, above, pos=0.5] {...} (n_7);
      \draw[->] (n_5) -- node[midway, above, pos=0.5] {$\epsilon$} (n_8);
      \draw[->] (n_6) -- node[midway, above, pos=0.5] {$\epsilon$} (n_8);
      \draw[->] (n_7) -- node[midway, above, pos=0.5] {$\epsilon$} (n_8);
    \end{tikzpicture}
    \caption{Localview graph for the \texttt{receive} keyword}
    \label{grafo:receive}
\end{figure}


\begin{figure}[ht!]
    \centering
    \begin{tikzpicture}[node distance={50mm}, thick, main/.style = {draw, circle}] 
      \node[state] (n_1) {1};
      \node[state] (n_2) [right of=n_1] {2};
      
      \draw[->] (n_1) -- node[midway, above, pos=0.5] {ProcSent ! message} (n_2);
    \end{tikzpicture}
    \caption{Localview graph for \texttt{!} keyword}
    \label{grafo:send}
\end{figure}

\begin{figure}[ht!]
    \centering
    \begin{tikzpicture}[node distance={40mm}, thick, main/.style = {draw, circle}] 
      \node[state] (n_1) {1};
      \node[state] (n_2) [right of=n_1] {2};
      
      \draw[->] (n_1) -- node[midway, above, pos=0.5] {$\Delta$ process} (n_2);
    \end{tikzpicture}
    \caption{Localview graph for the \texttt{spawn} function}
    \label{grafo:spawn}
\end{figure}

\paragraph{Recursive calls}
In the case of \textit{recursive} calls, a transition $\epsilon$ is created from the last  
node generated to the first node, as shown in Figure \ref{grafo:ricors}.
For now, further evaluation of the function is blocked.

\begin{figure}[ht!]
    \centering
    \begin{tikzpicture}[node distance={3cm}, thick, main/.style = {draw, circle}] 
      \node[state] (n_1) {1};
      \node[state] (n_2) [right of=n_1] {\ldots};
      \node[state] (n_3) [right of=n_2] {$n$};
      
      \draw[->] (n_1) -- node[midway, above, pos=0.5] {\ldots} (n_2);
      \draw[->] (n_2) -- node[midway, above, pos=0.5] {\ldots} (n_3);
      \draw[->] (n_3) to [out=120,in=60] node[midway, above, pos=0.5] {$\epsilon$} (n_1);
    \end{tikzpicture}
    \caption{Localview graph for a recursive call}
    \label{grafo:ricors}
\end{figure}

\paragraph{Function call}
When a call to an unknown function is encountered, the algorithm will create the local  
view of the called function and "connect" the beginning of the graph of the called  
function with the last state created in the local view of the calling function, linking  
them with an $\epsilon$ transition. The local view will continue from the last vertex of  
the call graph.

\begin{figure}[ht!]
    \centering
    \begin{tikzpicture}[node distance={2cm}, thick, main/.style = {draw, circle}] 
      \node[state] (n_1) {$1_a$};
      \node[state] (n_2) [right of=n_1] {\ldots};
      \node[state] (n_3) [right of=n_2] {$n$};
      \node[state] (n_4) [right of=n_3] {$1_b$};
      \node[state] (n_5) [right of=n_4] {\ldots};
      
      \draw[->] (n_1) -- node[midway, above, pos=0.5] {\ldots} (n_2);
      \draw[->] (n_2) -- node[midway, above, pos=0.5] {\ldots} (n_3);
      \draw[->] (n_3) -- node[midway, above, pos=0.5] {$\epsilon$} (n_4);
      \draw[->] (n_4) -- node[midway, above, pos=0.5] {\ldots} (n_5);
    \end{tikzpicture}
    \caption{Localview graph of a function call}
    \label{grafo:funcall}
\end{figure}

In graph \ref{grafo:funcall}, state $1_a$ is the initial state of the calling function,  
and state $1_b$ represents the first state of the local view of the called function.

\paragraph{Global view}
For global views, in \texttt{spawn}, the actor performing the operation is specified to  
the left of the symbol, as shown in Figure \ref{grafo:globspawn}. A spawn will be  
directly inserted into the graph. Each process will also be numbered in case multiple  
actors are created for the same function.

\begin{figure}[ht!]
    \centering
    \begin{tikzpicture}[node distance={7cm}, thick, main/.style = {draw, circle}] 
      \node[state] (n_1) {1};
      \node[state] (n_2) [right of=n_1] {2};
      
      \draw[->] (n_1) -- node[midway, above, pos=0.5] {ProcSpawner$_n\Delta$ProcSpawned$_m$} (n_2);
    \end{tikzpicture}
    \caption{Global view graph for the \texttt{spawn} function}
    \label{grafo:globspawn}
\end{figure}

For message sending and receiving, during the simulation of actors, if two actors are  
found executing a compatible \textit{send} and \textit{receive}, a state will be added  
to the global view as shown in Figure \ref{grafo:sendrecv}. To be compatible, the  
receiving process must match the data recipient, and the pattern matching of the  
\textit{receive} must correspond to the sent data. Transitions for \textit{send} and  
\textit{receive} that are ``empty" (i.e., messages that are sent but not processed by a  
\textit{receive} in any process) will not be shown in the global automaton.

\begin{figure}[ht!]
    \centering
    \begin{tikzpicture}[node distance={8cm}, thick, main/.style = {draw, circle}] 
      \node[state] (n_1) {1};
      \node[state] (n_2) [right of=n_1] {2};
      
      \draw[->] (n_1) -- node[midway, above, pos=0.5] {ProcSender$_n\to$ ProcRecv$_m$ : message} (n_2);
    \end{tikzpicture}
    \caption{Global view graph for \texttt{receive} and \texttt{!} keywords}
    \label{grafo:sendrecv}
\end{figure}


\subsubsection{Main algorithms}
Execution is divided into 3 main phases:
\begin{enumerate}
    \item Initialization of the \texttt{db\_manager}, data structures, and extraction of preliminary information (handled by the \texttt{metadata.erl} module): possible actors are extracted from the \texttt{export} attribute, the number of \texttt{spawn} executions is counted, and the ASTs of all functions are saved in the \texttt{db\_manager}. Meanwhile, an initial evaluation of the program flow is performed by initializing the names of actors and saving the argument passing to the \texttt{spawn} functions.
    \item Creation of local views for all possible actors, i.e., all functions that appear in the \texttt{export} at the beginning of a program (obtained from the first phase).
    \item Creation of the global view starting from the program's starting point and combining the local views created in phase two.
\end{enumerate}


\paragraph{Module \texttt{local\_view.erl}}
In the module that creates local views, the main function is \texttt{eval\_codeline}. This function evaluates the individual line of code and its arguments. It also adds nodes to the local view graph in the case of communication constructs. The function creates a binding between a variable and its content, if evaluable. Additionally, it creates branches with $\epsilon$ edges in the case of \texttt{if} and \texttt{case}. Later, the $\epsilon$ transitions will be eliminated through graph minimization. There are also some evaluations of useful \textit{built-in} functions for communication, such as \texttt{self()} which returns the ID of its own process and \texttt{register} which registers a process ID in an \textit{atom}.
Below is the pseudocode of some interesting branches.

\textbf{N.B.}: In Erlang, it is rare to specify from which process you want to receive a particular message because it is not known in advance which actors are present. Therefore, the global view will use the label \texttt{receive msg} to express the branch where a specific message is received. Pattern matching will also be specified, evaluating where possible, to facilitate the creation of the global view.

In section \ref{sec:corrisp}, the correspondence between code and graph for constructs that modify the local view will be explicitly stated. Instead, the branches corresponding to data types or \textit{built-in} functions (like \texttt{self()}) return a data structure representing the data, which will either be bound to a variable in the \texttt{match} branch or passed as an argument to a function.

\paragraph{Module \texttt{global\_view.erl}}

While creating a local view simply involves following the function code line by line, in the global view, we need to compose the local views while taking into account the various actors created. Therefore, an approximate execution will be simulated starting from the startup function. Each actor will be associated with a process of the \texttt{proc\_loop} function, which keeps track of available branches and the state of the process. This function is responsible for providing the main process with some information related to the actor.

\begin{lstlisting}[language=Erlang, caption=Pseudocode of \texttt{proc\_loop} function]
proc_loop(LocalView, CurrentState, MarkedEdges) ->
    receive
        {use_transition, Edge} ->
            % to avoid infinite loops
            NewState = verify_not_marked();
            proc_loop(LocalView, NewState, MarkedEdges);
        {P, get_info} ->
            P ! {someinfo},
            proc_loop(LocalView, CurrentState, MarkedEdges);
        stop -> terminated
    end.
\end{lstlisting}

Furthermore, messages could travel either within the same virtual machine or across the network, so message exchange occurs asynchronously. That is, if two sends, $A$ and $B$, are made to the same process, either $A$ or $B$ could arrive first, completely changing the execution flow of the program.

In the following code \ref{code:global}, the main function of the \texttt{global\_view.erl} module is presented, which creates the global view by combining actors and their local views. The basic idea is to create a \textit{depth-first view} (DFS) of the actor automata, stopping appropriately according to the communication rules.

\begin{lstlisting}[language=Erlang, caption=Main function for globalview construction, label=code:global]
progress_branches(BranchList) ->
    NewBranchList = lists:foreach(
        fun(Item) -> progress_single_branch(Item) end,
        BranchList
    ),
    progress_branches(NewBranchList).
progress_single_branch(Data) ->
    SendList = lists:foreach(
        fun(Name) -> eval_proc_until_send(Name) end,
        Data.proc_list
    ),
    lists:foreach(
        fun(SendData) -> 
            manage_send(duplicate_branch(SendData))
        end,
        SendList
    ).
\end{lstlisting}

The strategy for composing local views is to have all actors continue until any \textit{send} (line 3 of code \ref{code:global}). While searching for and finding the first \textit{send}, \textit{spawn} and \textit{receive} are also searched. If a \textit{spawn} is encountered, the corresponding node is immediately added to the graph, and the actor is created. If a \textit{receive} is encountered, the message queue of the process is checked. If a compatible message is found, a transition is created on the graph, and the two actors continue execution.

After being blocked on \textit{send} or \textit{receive}, processes will be duplicated on different branches for different executions (line 14 of code \ref{code:global}). Each "branch" of execution will differ depending on which \textit{send} is evaluated first. At the same time, a check is performed to see if a process is available to receive that message. If so, a transition is created on the global graph, and the actors continue. Otherwise, the "vacant" \textit{send} is inserted into an appropriate data structure.

The algorithm in \ref{code:global} is executed recursively until a relevant data structure is modified (such as the graph). On the first iteration that does not modify any data structure, the algorithm stops. The final graph will represent the communication that occurred asynchronously for messages sent from different actors. Examples of "branching" between \textit{send}s are shown in the example section.

Each module, after creating their respective automata, will be responsible for converting them into DOT language and saving them to a file. To visualize the graphs, simply copy the contents of the file into an application that interprets the DOT format.


\paragraph{Old state}

The tool can accurately parse the main constructs and features of the Erlang 
language. In Table \ref{erl}, we outline the current status: what is supported, 
what is not supported, and what are the next objectives. It was decided not to 
support certain constructs because they are not useful at this stage of the 
project. For example, supporting error handling with try-catch is unnecessary 
when variable passing is not yet implemented. Additionally, some constructs were
excluded because they are orthogonal to others, such as complex data structures. 
The tool generates a local view that closely aligns with the behavior of the 
original actor, and the algorithm for combining local views into a global view
performs well on simple examples. However, at present, it does not provide 
automatic error detection for certain properties in the global choreography.

% \begin{table}[!ht]
%     \centering
%     \begin{minipage}{0.49\textwidth}
%     \centering
%     \begin{tabular}{|c||c|}
%     \hline \textbf{Keyword} & \textbf{Support}    \\  \hline
%      atom & \cellcolor{ashgrey}yes    \\  \hline
%      integer & \cellcolor{ashgrey}yes \\   \hline
%      float &\cellcolor{ashgrey}yes    \\   \hline
%      boolean &\cellcolor{ashgrey}yes    \\   \hline
%      tuple & \cellcolor{ashgrey}yes    \\   \hline
%      list &\cellcolor{ashgrey}yes    \\   \hline
%      record & \cellcolor{red!50}no   \\   \hline
%      map & \cellcolor{red!50}no   \\   \hline
%      binary & \cellcolor{red!50}no    \\   \hline
%      if & \cellcolor{ashgrey}yes    \\   \hline
%      case & \cellcolor{ashgrey}yes    \\   \hline
%      receive &\cellcolor{ashgrey}yes    \\   \hline
%     \end{tabular}
%     \end{minipage}
%     \centering
%     \begin{minipage}{0.49\textwidth}
%     \begin{tabular}{|c||c|}
%     \hline \textbf{Keyword} & \textbf{Support}    \\  \hline
%      ! & \cellcolor{ashgrey}yes    \\   \hline
%      assignment &\cellcolor{ashgrey}yes    \\   \hline
%      function & \cellcolor{green!50}goal  \\   \hline
%      recursion & \cellcolor{green!50}goal   \\   \hline
%      hof &\cellcolor{green!50}goal    \\   \hline
%      when & \cellcolor{red!50}no    \\   \hline
%      self &\cellcolor{ashgrey}yes    \\   \hline
%      spawn &\cellcolor{ashgrey}yes    \\   \hline
%      rand:uniform &\cellcolor{ashgrey}yes    \\   \hline
%      try catch & \cellcolor{red!50}no    \\   \hline
%      after & \cellcolor{red!50}no    \\   \hline
%      math operation &\cellcolor{red!50}no    \\   \hline
%     \end{tabular} 
%     \end{minipage}
    
%     \caption{Supported constructs: Keywords in gray are already supported, keywords in red are not supported and are unlikely to be supported soon. Constructs in green are related to functions, because parameter passing is not supported.}
%     \label{erl}
% \end{table}