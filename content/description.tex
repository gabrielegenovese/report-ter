\paragraph{How to use the tool}
The easiest way to use the tool is from the command line interface (CLI)
using the help of \texttt{rebar3}, a standard build tool that provides various features
such as package management for community-created libraries, compilation, and
automated project testing. By cloning the project from the public GitHub
repository and running the \texttt{rebar3 escriptize} command in the main
directory, \texttt{rebar3} will, in this order, check for and download
dependencies if needed, compile the project. After
that, an executable can be used in \texttt{./\_build/default/bin/chorer}.

\begin{lstlisting}[caption=Usage message]
Usage:
  chorer <input> <entrypoint> <output> <ming> <gstate> <minl>

Extract a choreography automata of an Erlang program.

Arguments:
  input      Erlang source file (string)
  entrypoint Entrypoint of the program (atom)
  output     Output directory for the generated dot files (string), default: ./
  ming       Minimize the globalviews , default: false
  gstate     Global state are formed with previous messages , default: true
  minl       Minimize the localviews , default: true
\end{lstlisting}

\noindent The mandatory arguments are:
\begin{itemize}
    \item \textbf{Input}: The relative path string of the input Erlang program,
    from which the tool will generate local and global views.
    \item \textbf{Entrypoint}: The atom representing the function where the
    execution of the input program begins. This parameter is essential because
    Erlang does not have a conventional entry point function (i.e. the main in C,
    but in Erlang can be every exported function). It will be passed to the
    function that creates the global view to start the simulation.
\end{itemize}

\noindent The optional arguments are:
\begin{itemize}
    \item \textbf{Output}: The relative path string of the output folder where
    the local and global view files will be saved. Local views will be named
    \texttt{[function name with arity]\_local\_view.dot}, while the global view
    file will be named \texttt{[Entrypoint]\_global\_view.dot}.
    \item \textbf{Options}: Additional arguments define specific operations  
    to perform on either the local or global view. For instance, they  
    can be used to generate an optimized or minimized version of a  
    graph as output. These options allow for greater flexibility in  
    refining and analyzing the extracted views, enabling different 
    meaningful representations.
\end{itemize}

\subsubsection{From Erlang to Choreographies}
\label{sec:corrisp}

\paragraph{Local view}

The code of a \texttt{receive} operation corresponds to the Figure \ref{grafo:receive}:
each branch will be evaluated recursively, continuing the construction from the 
branches. Finally, all branches will merge into a common node with $\epsilon$ 
transitions, from which the evaluation of the local view will resume.  
The send operation (\texttt{Pid ! message}) operation corresponds to Figure \ref{grafo:send}.  
The \texttt{spawn} function call corresponds to Figure \ref{grafo:spawn}.  

\begin{figure}[!ht]
    \centering
    \begin{tikzpicture}[node distance={29mm}, thick, main/.style = {draw, circle}] 
      \node[state] (n_1) {1};
      \node[state] (n_2) [above right of=n_1] {2};
      \node[state] (n_4) [below right of=n_1] {$n$};
      \node[state] (n_3) [below=1.1\distance cm of n_2] {\ldots};
      \node[state] (n_5) [right of=n_2] {\ldots};
      \node[state] (n_6) [right of=n_3] {\ldots};
      \node[state] (n_7) [right of=n_4] {$m$};
      \node[state] (n_8) [below right of=n_5] {$m+1$};
      
      \draw[->] (n_1) -- node[midway, above left, pos=0.6] {0\#receive message0} (n_2);
      \draw[->] (n_1) -- node[midway, above, pos=0.5] {...} (n_3);
      \draw[->] (n_1) -- node[midway, below left, pos=0.6] {N\#receive messageN} (n_4);
      \draw[->] (n_2) -- node[midway, above, pos=0.5] {...} (n_5);
      \draw[->] (n_3) -- node[midway, above, pos=0.5] {...} (n_6);
      \draw[->] (n_4) -- node[midway, above, pos=0.5] {...} (n_7);
      \draw[->] (n_5) -- node[midway, above, pos=0.5] {$\epsilon$} (n_8);
      \draw[->] (n_6) -- node[midway, above, pos=0.5] {$\epsilon$} (n_8);
      \draw[->] (n_7) -- node[midway, above, pos=0.5] {$\epsilon$} (n_8);
    \end{tikzpicture}
    \caption{Localview graph for the \texttt{receive} keyword}
    \label{grafo:receive}
\end{figure}


\begin{figure}[!ht]
    \centering
    \begin{tikzpicture}[node distance={50mm}, thick, main/.style = {draw, circle}] 
      \node[state] (n_1) {1};
      \node[state] (n_2) [right of=n_1] {2};
      
      \draw[->] (n_1) -- node[midway, above, pos=0.5] {ProcSent ! message} (n_2);
    \end{tikzpicture}
    \caption{Localview graph for \texttt{!} keyword}
    \label{grafo:send}
\end{figure}

\begin{figure}[!ht]
    \centering
    \begin{tikzpicture}[node distance={40mm}, thick, main/.style = {draw, circle}] 
      \node[state] (n_1) {1};
      \node[state] (n_2) [right of=n_1] {2};
      
      \draw[->] (n_1) -- node[midway, above, pos=0.5] {$\Delta$ process} (n_2);
    \end{tikzpicture}
    \caption{Localview graph for the \texttt{spawn} function}
    \label{grafo:spawn}
\end{figure}

\paragraph{Recursive calls}
In the case of \textit{recursive} calls, a transition $\epsilon$ is created from the last  
node generated to the first node, as shown in Figure \ref{grafo:ricors}.
For now, further evaluation of the function is blocked.

\begin{figure}[!ht]
    \centering
    \begin{tikzpicture}[node distance={3cm}, thick, main/.style = {draw, circle}] 
      \node[state] (n_1) {1};
      \node[state] (n_2) [right of=n_1] {\ldots};
      \node[state] (n_3) [right of=n_2] {$n$};
      
      \draw[->] (n_1) -- node[midway, above, pos=0.5] {\ldots} (n_2);
      \draw[->] (n_2) -- node[midway, above, pos=0.5] {\ldots} (n_3);
      \draw[->] (n_3) to [out=120,in=60] node[midway, above, pos=0.5] {$\epsilon$} (n_1);
    \end{tikzpicture}
    \caption{Localview graph for a recursive call}
    \label{grafo:ricors}
\end{figure}

\paragraph{Function call}
When a call to an unknown function is encountered, the algorithm will create the local  
view of the called function and "connect" the beginning of the graph of the called  
function with the last state created in the local view of the calling function, linking  
them with an $\epsilon$ transition. The local view will continue from the last vertex of  
the call graph.

\begin{figure}[!ht]
    \centering
    \begin{tikzpicture}[node distance={2cm}, thick, main/.style = {draw, circle}] 
      \node[state] (n_1) {$1_a$};
      \node[state] (n_2) [right of=n_1] {\ldots};
      \node[state] (n_3) [right of=n_2] {$n$};
      \node[state] (n_4) [right of=n_3] {$1_b$};
      \node[state] (n_5) [right of=n_4] {\ldots};
      
      \draw[->] (n_1) -- node[midway, above, pos=0.5] {\ldots} (n_2);
      \draw[->] (n_2) -- node[midway, above, pos=0.5] {\ldots} (n_3);
      \draw[->] (n_3) -- node[midway, above, pos=0.5] {$\epsilon$} (n_4);
      \draw[->] (n_4) -- node[midway, above, pos=0.5] {\ldots} (n_5);
    \end{tikzpicture}
    \caption{Localview graph of a function call}
    \label{grafo:funcall}
\end{figure}

In graph \ref{grafo:funcall}, state $1_a$ is the initial state of the calling function,  
and state $1_b$ represents the first state of the local view of the called function.

\paragraph{Global view}
For global views, in \texttt{spawn}, the actor performing the operation is specified to  
the left of the symbol, as shown in Figure \ref{grafo:globspawn}. A spawn will be  
directly inserted into the graph. Each process will also be numbered in case multiple  
actors are created for the same function.

\begin{figure}[!ht]
    \centering
    \begin{tikzpicture}[node distance={7cm}, thick, main/.style = {draw, circle}] 
      \node[state] (n_1) {1};
      \node[state] (n_2) [right of=n_1] {2};
      
      \draw[->] (n_1) -- node[midway, above, pos=0.5] {ProcSpawner$_n\Delta$ProcSpawned$_m$} (n_2);
    \end{tikzpicture}
    \caption{Global view graph for the \texttt{spawn} function}
    \label{grafo:globspawn}
\end{figure}

For message sending and receiving, during the simulation of actors, if two actors are  
found executing a compatible \textit{send} and \textit{receive}, a state will be added  
to the global view as shown in Figure \ref{grafo:sendrecv}. To be compatible, the  
receiving process must match the data recipient, and the pattern matching of the  
\textit{receive} must correspond to the sent data. Transitions for \textit{send} and  
\textit{receive} that are ``empty" (i.e., messages that are sent but not processed by a  
\textit{receive} in any process) will not be shown in the global automaton.

\begin{figure}[!ht]
    \centering
    \begin{tikzpicture}[node distance={8cm}, thick, main/.style = {draw, circle}] 
      \node[state] (n_1) {1};
      \node[state] (n_2) [right of=n_1] {2};
      
      \draw[->] (n_1) -- node[midway, above, pos=0.5] {ProcSender$_n\to$ ProcRecv$_m$ : message} (n_2);
    \end{tikzpicture}
    \caption{Global view graph for \texttt{receive} and \texttt{!} keywords}
    \label{grafo:sendrecv}
\end{figure}


\subsubsection{Main algorithms}
\noindent The execution is divided into four main phases:
\begin{enumerate}
\item Initialization of the \texttt{db\_manager}, data structures, and
extraction of preliminary information (handled by the
\texttt{md.erl} module). Possible actors are extracted from
the \texttt{export} attribute, the number of \texttt{spawn}
executions is counted, and the ASTs of all functions are saved in
the \texttt{db\_manager}. Meanwhile, an initial evaluation of the
program flow is performed by initializing the names of actors and
saving the argument passing to the \texttt{spawn} functions.

\item Creation of local views for all possible actors, i.e., all functions
that appear in the \texttt{export} at the beginning of a program
(obtained from the first phase). This phase essentially acts as a compiler, 
translating Erlang programs into local choreographies.

\item Creation of the global view starting from the program's starting
point and combining the local views created in phase two.

\item Post-processing of the local and global view, inferring information on the
graph and extracting data useful for the benchmarks
\end{enumerate}

While creating a local view follows the function code line by line, the global  
view must compose local views while considering the created actors. An  
approximate execution is simulated from the startup function using the actor  
philosophy. Each actor has a manager process that tracks available transitions  
and the process state (i.e., local variables). This function provides the main  
process of the global view with actor-related information. Messages may travel  
within the same virtual machine or across a network, making message exchange  
asynchronous. If two sends target the same process, their order  
is unpredictable, altering the program flow.  

The global view creation algorithm follows an overapproximation approach,  
combining local views while dynamically creating communicating actors and  
exchanging messages. It explores all possible executions by handling forks in  
local views and evaluating message reception. A branch duplicates and takes  
another path in three cases: encountering a fork in a local view, multiple  
processes ready to receive, and evaluating message reception within a process.  
This ensures a thorough exploration of execution paths.  

The strategy for composing local views is to have all actors continue until  
any \textit{receive} is encountered. While searching for the first \textit{receive},  
\textit{spawn} and \textit{send} operations are also checked. If a \textit{spawn}  
occurs, the node is added to the graph, and the actor is created. If a  
\textit{send} occurs, the message is enqueued in the target process' message  
queue. After blocking on a \textit{receive}, possible messages are processed,  
creating a new execution path for each, adding new states to the global graph.  

The algorithm runs recursively until a relevant state change occurs. When an  
iteration fails to modify any actor's state, the execution stops. The final graph  
represents the asynchronous communication among actors for messages exchanged.  
