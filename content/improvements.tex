One key goal of the tool is to generate correct graph outputs, ensuring an  
accurate representation of program behavior. While refining abstract aspects  
of the project, several features, improvements, and bug fixes have been added  
to enhance the precision and consistency of the global view specification. The  
iterative process of refining the tool aims to minimize errors, and provide a
more reliable emulation of  actor-based execution. Future work will continue to
focus on refining these aspects to offer greater 
flexibility while maintaining correctness of the analyzed program.

\paragraph{Features}
Two key features have been added to the tool. The first one improves the local  
view module. The tool can now correctly parse programs where values are passed   
as arguments to functions, allowing for a more precise representation of  
data flow between processes. 
Previously, only functions without value passing were supported. This was a
significant limitation and, in fact, one of the highest-priority features.
It was the final major feature of a programming language to be added, and it is 
now included in the parsing module, enabling a broader range of program inputs.

The second feature added to the tool enhances over-approximation
in the global view module. During computation, the value of 
some data can become everything because we cannot know the outcome of an 
operation performed on the data (i.e. we don't support mathematical operation). 
The tool now substitutes this unknown value with the \texttt{ANY} keyword to 
indicate that it can represent any kind of data. If a message is exchanged with 
\texttt{ANY} as its content, it will match every possible receive branch,
ensuring a more over-approximated analysis of the program.

\paragraph{General improvements}
Some general improvements have been made on the tool. 
The tool is now more resilient to unexpected errors, reducing the  
likelihood of crashes and ensuring that it provides meaningful output in  
most cases, even when encountering invalid input. This was achieved through the
implementation of more error checks and recovery patterns within the codebase, 
where there were none before.

Also, a new library has been integrated for command-line argument management,  
making the CLI more robust and user-friendly, with better parsing of  
parameters and improved error messages.

Some effort has been put to improve the testing environment and benchmark 
generation. The tool now produces useful benchmarking information, which is 
processed by a dedicated Python script. This script collects and analyzes the 
data to generate meaningful performance benchmarks. Additionally, if the  
\texttt{correct\_gv.dot} file is present (that is a dot file containing 
the expected global view) the script performs an automatic correctness check. 
This feature lays the groundwork for more expressive and rigorous testing in 
the future, enabling better validation of the tool's accuracy and performance
across different scenarios and use cases.

\paragraph{Bug fixes}
Some notable bug fix improvements were made within the global view module that 
are needed to be mentioned. It has been resolved an issue where the tool
incorrectly printed warnings about failing to find the correct process during
the sending of a message. This bug caused the useless show of some false 
positive warnings. 

It also has been addressed a problem that prevented the exploration of certain
branches during the global view emulation, ensuring a more comprehensive
traversal of all possible execution paths. This bug prevented a full branch
exploration in certain programs. 

It has been fixed a bug that was causing excessive duplicate branches. 
This patch corrected an issue where unnecessary duplicate branches were created  
after receiving a message and defining a new variable. This fix improves  
the accuracy of the global view by reducing redundant computations.  
