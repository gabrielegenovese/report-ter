\subsubsection{Attributed grammar}
At the beginning of this period, it was challenging to clearly define the role
of the parser. There was no formal specification outlining how the tool should
process a program during its initial analysis. This lack of clarity led to
difficulties in understanding the expected behavior and designing a structured
approach for parsing.
To address this, we first examined the tool's existing implementation and
identified key areas requiring specification. Establishing a well-defined set
of rules became essential for ensuring consistent behavior and improving the
accuracy of the parsing process. Through iterative refinement, we aimed to
create a structured framework that guides the tool in extracting meaningful
information from the program.
Therefore, to better
understand how the parsing of a file and the creation of a localview 
should work, we created an attributed grammar for a subset of the Erlang 
language. An attributed grammar extends a context-free grammar by associating 
attributes with its symbols and defining semantic rules. Attributes can be 
\textit{synthesized} (computed from child nodes) or \textit{inherited}
(passed from parent nodes). 

The attribute part can be seen as a set of operations that the compiler  
performs during its analysis. This serves two key purposes: it provides  
a clear understanding of what the tool can parse from the programming  
language, and it defines how information propagates through the parse  
tree. This propagation helps identify the core data structures that  
underpin the tool’s functionality.  

The following attributed grammar will be presented alongside comments  
to provide insights into the tool’s behavior. First, we introduce the  
grammar fragment, followed by the corresponding operations applied  
to the local view data structure. These operations illustrate how the  
tool processes and interprets parsed elements, highlighting its  
internal mechanisms. 

The \texttt{link} function is used to establish a logical connection  
between two nodes. Its third argument defines the label of the transition.  
If this argument is not provided, the transition is considered an  
$\epsilon$-transition, representing an unlabeled connection. 

\paragraph{Program} For simplicity, we consider an Erlang program as a set 
of one or more functions, that can be called by the user. All the needed 
function are within the program.

\bigskip

\noindent $prog \to (function)^+$

\bigskip

A complete program does not have a local view, so there is no need  
to define attributes.  

\paragraph{Function}  
A function can have its own local view representation.  
In Erlang, pattern matching allows multiple definitions,  
so we must distinguish between the base case and the  
pattern matching case. The base case is straightforward,  
so we only present the pattern matching case.  

\bigskip

\noindent Base case: $function \to fun.$

\noindent With multiple definitions: $function \to fun_1;...;fun_N.$

\begin{verbatim}
function.nodes = new U fun1.nodes U ... U funN.nodes
function.edges = link(new,fun1.first) U ... U 
                 link(new,funN.first) U 
                 fun1.edges U ... U funN.edges
function.first = new
\end{verbatim}

Each definition has its own local view representation, so we must link  
each one to a new state. This state serves as the starting point of  
the entire local view. The nodes consist of all other nodes, and  
the same applies to the edges.  

\paragraph{Function body}  
The body of a function consists of a name (which is an atom),  
a set of arguments (which may be empty), and a list of expressions.  

\bigskip

\noindent $fun \to Atom(X_1,...,X_n) \to Exprs$

\begin{verbatim}
fun.nodes = Exprs.nodes
fun.edges = Exprs.edges
fun.first = Exprs.first
fun.last = Exprs.last
fun.context = [ X1 -> Param[1], ..., Xn -> Param[n] ]
fun.ret_var = Exprs.ret_var
\end{verbatim}

When encountering a function, we simply add its arguments to the context.  
The \texttt{Param} is taken as input, with a default value of \texttt{ANYDATA}.  
Additionally, we can perform semantic checks to verify the function’s  
existence.  

\paragraph{Expressions}  
A list of expressions can have one or more elements. The single-expression  
case is straightforward, so its attributes are omitted.  

\bigskip

\noindent Base case: $Exprs \to expr$

\noindent With multiple expression: $Exprs \to expr,Exprs'$

\begin{verbatim}
Exprs.nodes = expr.nodes U Exprs'.nodes
Exprs.edges = expr.edges 
              U Exprs'.edges 
              U link(expr.last, Exprs'.first)
Exprs.first = expr.first
Exprs.last = Exprs'.last
Exprs.context = expr.context U Exprs'.context
Exprs.ret_var = Exprs'.ret_var
\end{verbatim}

Expressions represent the lines of code in an Erlang program.  
We cover only the essential ones (e.g., assignments, function calls)  
and those related to communication.  

\paragraph{Send}   
As mentioned earlier, a send should add a transition to the local graph.  

\bigskip

\noindent $expr \to expr'\ \ !\ \ expr''$

\begin{verbatim}
expr.nodes = expr'.nodes U expr''.nodes U new1 U new2
expr.edges = expr'.edges U expr''.edges
             U link(expr'.last, expr''.first)
             U link(expr''.last, new1)
             U link(new1, new2, expr'.ret_var 
                  + " ! " 
                  + expr''.ret_var)
expr.first = expr'.first
expr.last = new2
expr.context = expr'.context U expr''.context
expr.ret_var = expr''.ret_var
\end{verbatim}

In Erlang, both the left-hand side and right-hand side of an operation  
can be expressions. First, the left-hand side expression is evaluated,  
followed by the right-hand side. Finally, a send edge is inserted with  
the corresponding variable as the label. The left-hand side must contain  
a process identifier.  

\paragraph{Receive}  
Like for a send, a receive operation should add a transition to  
the local graph for each matching pattern.  

\bigskip

\noindent $expr \to receive patter_1 \to Exprs_1; ...; pattern_n \to Exprs_n end$

\begin{verbatim}
expr.first = new1
expr.last = new2
expr.nodes = Exprs1.nodes U ... U Exprsn.nodes U new1 U new2
expr.edges = Exprs1.edges U ... U Exprsn.edges
             U link(new1, Exprs1.first) 
             U ...
             U link(new1, Exprsn.first)
             U link(Exprs1.last, new2)
             U ...
             U link(Exprsn.last, new2, espilon)
\end{verbatim}

A \texttt{receive} operation can contain an entire block of code 
that must be evaluated.  
Its return data cannot be determined statically, as it varies depending  
on the matching pattern during the global view phase. Additionally,  
since the context is modified during global view simulation,  
we leave these two attributes unchanged.  

\paragraph{Spawn call}  
The same reasoning used before applies to the \texttt{spawn} function.  

\bigskip

\noindent $expr \to spawn(Atom, Params)$

\begin{verbatim}
expr.nodes = new1 U new2
expr.edges = link(new1, new2, "spawn Atom")
expr.first = new1
expr.last = new2
expr.ret_var = newVar(type: pid, value: random) 
\end{verbatim}

The \texttt{spawn} function returns a new variable as output.  
Its type should be \texttt{pid}, with no assigned name.  
An additional logical identifier is included to differentiate  
it during global view construction. The context is left untouched
as it's not modified.

\paragraph{Recursive call}  
Recursive calls must be considered, as they are fundamental  
in functional languages like Erlang.  

\bigskip

\noindent $expr \to FunName(Params)$

\begin{verbatim}
expr.edges = expr.edges U link(FunName.first, expr.last)
expr.ret_var = null
\end{verbatim}

For recursive calls, the last added node must be linked to the first  
node of the local view being created. Using the function name,  
we can retrieve its previously created first node.  
Since a single analysis cannot determine the function's return type,  
we set the return variable to \texttt{null}. After evaluating the  
recursive call, the analysis stops, allowing only tail-recursive  
functions.  

\paragraph{Generic function call}  
To complete the previous analysis, we define attributes for  
standard function calls.  

\bigskip

\noindent $expr \to Atom(Params)$

\begin{verbatim}
G = get_localview(Atom, Param)
expr.nodes = G.nodes
expr.edges = G.edges
expr.first = G.first
expr.last = G.last
expr.ret_var = G.ret_var 
\end{verbatim}

With \texttt{get\_localview}, we retrieve the necessary information  
to complete the local view for a generic function call.  

\paragraph{Assignment}  
Finally, assignments must be handled to support basic programs  
with variables. Assignments in Erlang are peculiar as it has
immutable variables: they behave  
as pattern matching when a variable is already defined.
Thus, semantic checks can be performed.  

\bigskip

\noindent $expr \to pattern = expr'$

\begin{verbatim}
expr.nodes = expr'.nodes
expr.edges = expr'.edges
expr.first = expr'.first
expr.last = expr'.last
expr.context = [pattern -> expr'.ret_var]
\end{verbatim}

The pattern should add the new variable to the context  
with its assigned name.  

\paragraph{Conclusion}
We defined the attributed grammar for the \texttt{localview} module of our tool.  
This process highlighted the need for a \texttt{localview} with the following  
attributes: \texttt{nodes} and \texttt{edges}, with labels to represent the  
automaton; a \texttt{first} and \texttt{last} node to track entry and exit  
points; and a \texttt{context} along with \texttt{ret\_var} for basic  
variable management.
Some of these attributes are already present in the current codebase,  
but their handling differs, and additional arguments must be incorporated.  
Thus, a refactoring process is required to align the implementation with  
this attributed grammar.

This technique is particularly powerful in our case. 
For instance, a known issue with  
mutual recursive functions can be effectively addressed by improving  
context management inspired by the attributed grammar.
Using this method extensively could be key to  
efficiently representing \texttt{localviews}. 
However, the refactoring process is extensive and time-consuming.  
Therefore, it has been identified as a high-priority task for future work.  



\subsubsection{New examples}
In order to better illustrate our point, we present in this section two examples (using Erlang-like, actor-based pseudocode), each one
composed by a pseudocode and the corresponding Choreography
Automaton~\cite{coordination2020-chorAuto} (a graphical way of
representing a finite-state machine) of the global view of the
communicating system. Here, a \emph{global view} is an abstract description of
all the possible behaviors of the full system. We consider a language
where receive statements are blocking operations. A state where each
participant has completed its task and terminated is called \emph{final}. As such, a
state that is not final and has no outgoing transitions is a \emph{deadlock}.

The first example is a concise reproduction of the dining philosophers
problem, which highlights a possible deadlock. The second example
shows a possible mutual exclusion error 
when operating a simple bank account.
%in a database.
%
Both examples are available online~\cite{chorer_examples}; the
corresponding automata have been obtained with the help of our prototype tool
Chorer~\cite{chorer}.

\lstset{language=erlang, basicstyle=\sffamily\footnotesize,
  keywordstyle=\color{blue}, numberstyle=\tiny, numbers=none,
  showspaces=false, showstringspaces=false, frame=tL,
  backgroundcolor=\color{black!5}, morekeywords={send, to, from} }


In the examples, we exploit two operations for sending and receiving messages,
respectively.
%
More precisely, \lstinline[language=erlang, morekeywords={send,
  to}]{send msg to proc} sends message \lstinline{msg} to process
\lstinline{proc}, while \lstinline[morekeywords={receive,
  from}]{receive pat1 from proc1 -> e1;...;patn from procn -> en} 
  represents a branching point where the process receives 
  the first message 
  that matches a pattern $\mathsf{pati}$ and, then, 
  continues with the execution of $\mathsf{ei}$. 
  As in Erlang, 
  %we allow \lstinline[language=erlang]{receive}
%statement with multiple clauses where the 
pattern matching is tried from top to bottom.
% when there are
%multiple clauses.
When a receive has only one clause, we abbreviate it as
\lstinline[morekeywords={receive,from}]{receive pat from proc}
(and continues with the execution of the next sentence).
\paragraph{Dining Philosophers Example}
This example is taken from the \texttt{dining} case study from the benchmark 
suite of the tool.
Let us consider a program with two participants playing the role of a
dining philosopher who shares two forks with the other participant.

\begin{lstlisting}
philosopher(Fork1, Fork2) ->
  send req to Fork1,
  receive ack from Fork1,
  send req to Fork2,
  receive ack from Fork2,
  eat(),
  send release to Fork1,
  send release to Fork2,
  philosopher(Fork1, Fork2).

fork() ->
  receive req from Phil,
  send ack to Phil,
  receive release from Phil,
  fork().
\end{lstlisting}
%
% Consider the following 
% %\footnote{The
% %corresponding Erlang code can be found in our
% %\href{https://github.com/gabrielegenovese/chorer/blob/example/deadlock/examples/paper_example/deadlock/deadlock.erl}{github}
% %repository.} 
% program, where each function defines the behavior of a
% participant in the communicating system, which can be instantiated
% multiple times.  The program models two dining philosophers.
%

%
The behavior of the philosophers is given by the pseudocode on the
right while pseudocode for the behavior of the forks is discussed below.
Each philosopher first acquires the forks (starting with the one on
its right, that is the one with the same index), then eats (with the
function call \lstinline{eat()} representing some terminating local
computation), and finally releases the forks before recurring.
Parameters \lstinline{Fork1} and \lstinline{Fork2} are references to
processes executing the behavior of forks described by the pseudocode on the left
that repeatedly waits for the request from process \lstinline{Phil},
\lstinline{ack}s the request, and waits for the \lstinline{release}
message from \lstinline{Phil}.


There are two possible behaviors of the system. The first (good) one where the philosophers alternate eating infinitely and a second (bad) behavior 
%three
%possible executions of the system. In the first one, the first
%philosopher takes both forks, eats, then releases them,
%allowing the second philosopher to do the same. In the second
%execution, the philosophers eat in the opposite order. Given the recursive behavior of the \texttt{philosopher} 
%function, the actors could alternate infinitely.
%In the third possible execution, 
where both philosophers
manage to take only one fork each %, but they cannot take the other, as they
%are waiting for the other philosopher to release it. This 
resulting in a deadlock. 
Figure~\ref{graph:philosophers} depicts the global Choreography Automaton
representing the program above. 
We have two recursive executions where both philosophers
eat, represented as loops, and two executions which end in the same
deadlock state.
%\footnote{The full global view,
%obtained with our prototype tool Chorer \cite{chorer}, can be seen in
%our
%\href{https://github.com/gabrielegenovese/chorer/blob/example/deadlock/examples/paper_example/deadlock/main_0_global_view.dot}
  %   {github} repository.}.
%
%In the automaton, State 1 is the entry point. From State 1, we have
%two possible transitions, to State 2 and State 3 respectively, which
%model the requests (\texttt{req}) of the philosophers to the forks
%with the corresponding index. The two transitions can occur in any
%order (indeed there are transitions to State 5 closing the commuting
%diamond).
%
%Transitioning from State 1 to State 2 means that the system first 
%performs the \texttt{req} communication for the first philosopher. 
%Transitioning from State 1 to State 3 means that it first performs 
%the \texttt{req} communication for the second philosopher.
%States 2 and 3 represent states where there is a race between the
%philosopher that already has one fork taking the other (transitions on
%the side, leading to a loop), or the other philosopher taking the
%fork, leading to the deadlock in State 5. 
The deadlock is 
visible since State 5 is not final, and it has no outgoing transitions.
%Also, the system has a race, this is visible in the choreography
%automaton since State 3 has two outgoing transitions with the same
%receiver (the same for State 2), but we want to see these transitions
%since some of them lead to the deadlock state. 
  The automaton is simplified %w.r.t.~a full model of the program, 
  as it does not display the \texttt{ack} messages, and the
  \texttt{release} messages are merged since their order is
  irrelevant. The focus here is solely on the order of the
  \texttt{req} messages. One can imagine to first extract the full
  choreography automaton, and then merge equivalent behaviors and abstract away from uninteresting transitions.
%  to focus on the ones of interest. 
%  Another possible simplification
%  would be to notice that the system is symmetric (w.r.t.~the swap of
%  the two philosophers), hence it would be enough to analyze one of
%  the two halves.



\begin{figure}[t]
  \centering
  % \includegraphics[scale=.35]{images/deadlock.png}
  % \makebox[\textwidth][c]{
  \resizebox{0.9\textwidth}{!}{%
      \begin{tikzpicture}[node distance={40mm}, thick, main/.style = {draw, circle}] 
        \node[state] (n_1) {1};
        \node[state] (n_2) [below right of=n_1] {2};
        \node[state] (n_3) [below left of=n_1] {3};
        \node[state] (n_4) [right of=n_2] {4};
        \node[state] (n_5) [below of=n_1, fill=red] {5};
        \node[state] (n_6) [left of=n_3] {6};
        
        \draw[->] (n_1) -- node[midway, right, pos=0.5] {philo0→fork0:req} (n_2);
        \draw[->] (n_1) -- node[midway, left, pos=0.5] {philo1→fork1:req} (n_3);
        \draw[->] (n_2) -- node[midway, below right, pos=0.5] {philo1→fork1:req} (n_5);
        \draw[->] (n_3) -- node[midway, below left, pos=0.5] {philo0→fork0:req} (n_5);
        \draw[->] (n_2) -- node[midway, above, pos=0.5] {philo0→fork1:req} (n_4);
        \draw[->] (n_4) to[bend right=20] node[midway, right, pos=0.5] {philo0→fork\{0,1\}:release} (n_1);
        \draw[->] (n_3) -- node[midway, above, pos=0.5] {philo1→fork0:req} (n_6);
        \draw[->] (n_6) to[bend left=20] node[midway, left, pos=0.5] {philo1→fork\{0,1\}:release} (n_1);
      \end{tikzpicture}
    }
  % }%
  \caption{Global view of the dining philosophers example.}
  \label{graph:philosophers}
\end{figure}
This case study is taken from the \texttt{account} example from the benchmark 
suite of the tool.
We now consider a system where a bank account is accessed
by two clients, dubbed C1 and C2.
% 
\tikzset{
	hpath/.style={
			very thick,
			line cap = round,
			line join = round,
			line width=0.1cm,
			opacity=.70,
			color = teal!30
		}
}

\bigskip

\begin{lstlisting}
account(Value) ->
  receive
    read from Client ->
      send Value to Client,
      account(Value);
    NewValue from Client ->
      account(NewValue).

client() ->
  send read to Acc,
  receive Value from Acc,
  % operations on Value
  send NewValue to Acc.
\end{lstlisting}

\bigskip

The pseudocode yields the behavior of the bank account,
where $\mathsf{Value}$ represents its current balance.
This process waits for requests from a client.
A request can either be a \lstinline{read} access to know
the current balance or an update request of such value to a
\lstinline{NewValue}.

Symmetrically, client processes C1 and C2 behave according to the
pseudocode on the left: the process reads the current balance
from the account,
performs some internal operations based on such value, and
updates the balance.
%
The global view of the communicating system is depicted in
Figure~\ref{fig:account}.

\newcommand\dummy{C}
\begin{figure}[!ht]
	\centering
	\begin{tikzpicture}[node distance={27mm}, scale = .6, transform shape, thick, main/.style = {draw, circle}]
		\node (n_1) [state] {};
		\node (n_2) [state, below left of=n_1] {};
		\node (n_3) [state, below right of=n_1] {};
		\node (n_4) [state, left= 3.5cm of n_2] {4};
		\node (n_5) [state, below=3cm of n_1] {};
		\node (n_6) [state, right= 3.5cm of n_3] {6};
		\node (n_7) [state, below of=n_4] {7};
		\node (n_8) [state, below left of=n_5] {};
		\node (n_9) [state, below right of=n_5] {};
		\node (n_10) [state, below of=n_6] {10};
		\node (n_11) [state, accepting, below= of n_7] {11};
		\node (n_12) [state, accepting, below=1.5cm of n_8] {};
		\node (n_13) [state, accepting, below= 1.5cm of n_9] {};
		\node (n_14) [state, accepting, below=2.5cm of n_10] {14};

		\draw[->] (n_1) -- node[midway, above left] {acc→\dummy1:Value} (n_2);
		\draw[->] (n_2) -- node[midway, below left] {acc→\dummy2:Value} (n_5);
		\draw[->] (n_5) -- node[midway, above left=-2mm] {\dummy1→acc:NewValue} (n_8);
		\draw[->] (n_8) -- node[midway, below left] {\dummy2→acc:NewValue} (n_12);

		\draw[->] (n_1) -- node[midway, above right] {acc→\dummy2:Value} (n_3);
		\draw[->] (n_3) -- node[midway, below right] {acc→\dummy1:Value} (n_5);
		\draw[->] (n_5) -- node[midway, above right=-2mm] {\dummy2→acc:NewValue} (n_9);
		\draw[->] (n_9) -- node[midway, below right] {\dummy1→acc:NewValue} (n_13);

		\draw[hpath] ($(n_1.center) + (-5pt,-5pt)$) -- (n_2.center) -- (n_5.north west) -- ($(n_5.center)+(-5pt,0)$) -- (n_5.south west) -- (n_8.center) -- ($(n_12.center) + (0pt,5pt)$);
		\draw[hpath] ($(n_1.center) + (5pt,-5pt)$) -- (n_3.center) -- (n_5.north east)  -- ($(n_5.center)+(5pt,0)$) -- (n_5.south east)  -- (n_9.center) -- ($(n_13.center) + (0pt,5pt)$);

		\foreach \n in {1,2,3,5,8,9,12,13}{
				\node at (n_\n) {\n};
			}

		\draw[->] (n_2) -- node[midway, above=3mm] {C1→acc:NewValue} (n_4);
		\draw[->] (n_4) -- node[midway, above left] {acc→C2:Value} (n_7);
		\draw[->] (n_7) -- node[midway, above left] {C2→acc:NewValue} (n_11);

		\draw[->] (n_3) -- node[midway, above=3mm] {C2→acc:NewValue} (n_6);
		\draw[->] (n_6) -- node[midway, above right] {acc→C1:Value} (n_10);
		\draw[->] (n_10) -- node[midway, above right] {C1→acc:NewValue} (n_14);
	\end{tikzpicture}
	\caption{Global view of the bank account example}
	\label{fig:account}
\end{figure}

We can observe two correct executions where the operations are
performed in a read-update-read-update order (taking the path via
states 1-2-4-7-11 or the one via states 1-3-6-10-14).
%

However, there is also a read-read-update-update order on the
highlighted paths.
%
Although the choreography is not inherently incorrect, these
highlighted paths could represent a violation of mutual exclusion
which may be undesirable for developers in certain
contexts.
The choreography automaton in Figure~\ref{fig:account} helps in spotting
this issue.

%%% Local Variables:
%%% mode: LaTeX
%%% TeX-master: "main"
%%% End:


\subsubsection{Feature}
During this period, various features have been added to the tool:
\begin{itemize}  
    \item \textbf{Improved over-approximation in the global view}:  
    If, during computation, the value of some data is lost, the tool now  
    uses the \texttt{ANY} constant to indicate that it can represent any kind  
    of data. If a message is exchanged with \texttt{ANY} as its content, it  
    will match every possible receive branch, ensuring a more comprehensive  
    analysis.  
    
    \item \textbf{Basic value passing in the local view}:  
    The tool can now correctly parse programs where values are passed as  
    arguments to functions, allowing for a more accurate representation of  
    data flow between processes.  
    
    \item \textbf{Enhanced error handling}:  
    The tool is now more resilient to unexpected errors, reducing the  
    likelihood of crashes and ensuring that it provides meaningful output in  
    most cases, even when encountering invalid input.  
    
    \item \textbf{Improved CLI experience}:  
    A new library has been integrated for command-line argument management,  
    making the CLI more robust and user-friendly, with better parsing of  
    parameters and improved error messages.  

    \item \textbf{Improved testing environment and benchmark generation}:  
    The tool now produces useful benchmarking information, which is processed  
    by a dedicated Python script. This script collects and analyzes the data  
    to generate meaningful performance benchmarks. Additionally, if the  
    \texttt{correct\_gv.dot} file is present—containing the expected global  
    view—the script performs an automatic correctness check. This feature  
    lays the groundwork for more expressive and rigorous testing in the future,  
    enabling better validation of the tool's accuracy and performance across  
    different scenarios.  

\end{itemize}

\subsubsection{Bug Fix}
All of these improvements are made within the global view module:  
\begin{itemize}  
    \item \textbf{Fixed a bug causing false positive warnings}:  
    Resolved an issue where the tool incorrectly printed warnings about  
    failing to find the correct process during the emulation of a global view.  
    
    \item \textbf{Fixed a bug preventing full branch exploration}:  
    Addressed a problem that prevented the exploration of certain branches  
    during the global view emulation, ensuring a more comprehensive traversal  
    of all possible execution paths.  
    
    \item \textbf{Fixed a bug causing excessive duplicate branches}:  
    Corrected an issue where unnecessary duplicate branches were created  
    after receiving a message and defining a new variable. This fix improves  
    the accuracy of the global view by reducing redundant computations.  
\end{itemize}  


\subsubsection{Benchmarks}
\begin{table}[!ht]
\centering
\begin{tabular}{|c|c|c|c|c|c|c|}
\hline
Example & Lines & GV Nodes & GV Edges & Warnings & Errors & Runtime \\ 
\hline
account & 23 & 19 & 27 & 0 & 2 & 0.261s \\ 
dining & 31 & 29 & 44 & 0 & 2 & 0.250s \\ 
hello & 24 & 3 & 3 & 2 & 0 & 0.189s \\ 
async & 20 & 6 & 6 & 0 & 0 & 0.188s \\ 
ticktackstop & 46 & 12 & 19 & 7 & 0 & 0.210s \\ 
ticktackloop & 32 & 5 & 5 & 2 & 0 & 0.184s \\ 
customer & 54 & 12 & 16 & 1 & 0 & 0.209s \\ 
serverclient & 41 & 7 & 8 & 8 & 3 & 0.193s \\ 
trick & 24 & 8 & 8 & 0 & 0 & 0.185s \\ 
airline & 23 & 12 & 20 & 1 & 0 & 0.225s \\ 
conditional-case & 26 & 10 & 15 & 1 & 16 & 0.198s \\ 
for-loop-recursion & 18 & 8 & 9 & 0 & 0 & 0.186s \\ 
function-call & 17 & 3 & 3 & 1 & 2 & 0.184s \\ 
high-order-fun & 21 & 10 & 14 & 0 & 3 & 0.193s \\ 
if-cases & 57 & 86 & 134 & 185 & 30 & 0.543s \\ 
pass & 16 & 3 & 2 & 0 & 0 & 0.196s \\ 
producer & 30 & 8 & 7 & 0 & 1 & 0.198s \\ 
spawn & 22 & 9 & 8 & 0 & 0 & 0.182s \\ 
unknown & 13 & 1 & 1 & 0 & 0 & 0.187s \\ 
foo1 & 18 & 6 & 7 & 0 & 0 & 0.215s \\ 
foo2 & 23 & 4 & 3 & 1 & 1 & 0.190s \\ 
foo3 & 22 & 10 & 14 & 0 & 0 & 0.195s \\ 
foo4 & 20 & 12 & 15 & 0 & 2 & 0.194s \\ 
foo5 & 18 & 39 & 87 & 1 & 0 & 0.317s \\ 
foo6 & 24 & 6 & 7 & 15 & 2 & 0.200s \\ 
foo7 & 41 & 43 & 121 & 0 & 6 & 0.520s \\ 
foo8 & 29 & 27 & 95 & 0 & 171 & 2.893s \\ 
foo9 & 14 & 2 & 3 & 1 & 3 & 0.191s \\ 
foo9b & 21 & 4 & 4 & 14 & 1 & 0.188s \\ 
foo9c & 15 & 6 & 11 & 0 & 0 & 0.198s \\ 
foo9d & 16 & 3 & 2 & 0 & 0 & 0.186s \\ 
foo9e & 24 & 9 & 13 & 0 & 5 & 0.193s \\ 
foo9f & 25 & 4 & 6 & 0 & 4 & 0.193s \\ 
foo9g & 25 & 21 & 49 & 0 & 7 & 0.231s \\ 
foo9h & 23 & 12 & 26 & 0 & 5 & 0.196s \\ 
ping & 36 & 6 & 5 & 1 & 0 & 0.183s \\ 
airline & 33 & 18 & 34 & 1 & 0 & 0.225s \\ 
meViolation & 40 & 38 & 54 & 2 & 4 & 0.255s \\ 
purchase & 47 & 25 & 44 & 6 & 0 & 0.254s \\ 
\hline
\end{tabular}
\caption{Global view data}
\label{tab:gvbench}
\end{table}

