\subsubsection{Attributed grammar}
To better understand how the parsing of a file and the creation of a localview 
should work, we created an attributed grammar for a subset of the Erlang language.
An attributed grammar extends a context-free grammar by associating attributes with 
its symbols and defining semantic rules. Attributes can be \textit{synthesized} (computed 
from child nodes) or \textit{inherited} (passed from parent nodes). These rules define how 
information propagates through the parse tree. In the following attributed grammar, 
we define the localview of Chorer's constructs.

\paragraph{Program}

$prog \to (funPattern)^+$

A program doesn't have a local-view, no need to write attributes.

\paragraph{Function 
}
\paragraph{List}

$funList \to (funPattern)^+$

We create only the local-views that we need.

\paragraph{Pattern}

Base case: $funPattern \to fun.$ (trivial)

$funPattern \to fun_1;...;fun_N.$

\begin{verbatim}
funPattern.nodes = new U fun1.nodes U ... U funN.nodes
funPattern.edges = link(new,fun1.first) U ... U 
                   link(new,funN.first) U 
                   fun1.edges U ... U funN.edges
funPattern.first = new
\end{verbatim}

\paragraph{Body}

$fun \to Atom(X_1,...,X_n) \to Exprs$

Attributes:
\begin{verbatim}
fun.nodes = new U Exprs.nodes
fun.edges = Exprs.edges U link(new, E.first)
fun.first = new
fun.last = Exprs.last
fun.context = [ X1 -> Param[1], ..., Xn -> Param[n] ]
fun.ret_var = E.ret_var
\end{verbatim}

\texttt{Param} is taken as input (default to \texttt{ANYDATA})

\paragraph{Expressions}

\paragraph{Concatenation}

Base case: $Exprs \to expr$ (trivial)

$Exprs \to expr,Exprs'$

Attributes:
\begin{verbatim}
Exprs.nodes = expr.nodes U Exprs'.nodes
Exprs.edges = expr.edges 
              U Exprs'.edges 
              U link(expr.last, Exprs'.first, epsilon)
Exprs.first = expr.first
Exprs.last = Exprs'.last
Exprs.context = expr.context U Exprs'.context
Exprs.ret_var = Exprs'.ret_var
\end{verbatim}

\paragraph{Send}

$expr \to expr'\ \ !\ \ expr''$

Attributes:
\begin{verbatim}
expr.nodes = expr'.nodes U expr''.nodes U new1 U new2
expr.edges = expr'.edges U expr''.edges
             U link(expr''.last, new1, epsilon)
             U link(new1, new2, expr'.ret_var 
                + " ! " 
                + expr''.ret_var)
expr.first = expr'.first
expr.last = new2
expr.context = expr'.context U expr''.context
expr.ret_var = expr''.ret_var
\end{verbatim}

Firstly evaluate the left-hand side expression, then the right hand side one. 
Lastly insert the sends edge.

\paragraph{Receive}

$expr \to receive patter_1 \to Exprs_1; ...; pattern_n \to Exprs_n end$

Attributes:
\begin{verbatim}
expr.first = new1
expr.last = new2
expr.nodes = Exprs1.nodes U ... U Exprsn.nodes U new1 U new2
expr.edges = Exprs1.edges U ... U Exprsn.edges
             U link(new1, Exprs1.first, epsilon) 
             U ...
             U link(new1, Exprsn.first, epsilon)
             U link(Exprs1.last, new2, epsilon)
             U ...
             U link(Exprsn.last, new2, espilon)
expr.ret_var = ANYDATA (overapprox)
context?
\end{verbatim}

\paragraph{Spawn call}

$expr \to spawn(Atom, Params)$

Attributes:
\begin{verbatim}
expr.nodes = new1 U new2
expr.edges = link(new1, new2, "spawn Atom")
expr.first = new1
expr.last = new2
expr.ret_var = newVar(type: pid, value: random) 
\end{verbatim}

\paragraph{Recursive call}

$expr \to FunName(Params)$

Attributes:
\begin{verbatim}
expr.edges = expr.edges U link(FunName.first, expr.last)
expr.ret_var = null
\end{verbatim}

Use of inherited attributes
Stop evaluation

\paragraph{Generic function call} 

$expr \to Atom(Params)$

Attributes:
\begin{verbatim}
G = get_localview(Atom, Param)
expr.nodes = G.nodes
expr.edges = G.edges
expr.first = G.first
expr.last = G.last
expr.ret_var = G.ret_var 
\end{verbatim}

\paragraph{Assignment}

$expr \to pattern = expr'$

Attributes:
\begin{verbatim}
expr.nodes = expr'.nodes
expr.edges = expr'.edges
expr.first = expr'.first
expr.last = expr'.last
expr.context = [pattern -> expr'.ret_var]
\end{verbatim}

\subsubsection{New examples}
\input{content/example}
\paragraph{Dining Philosophers Example}
This example is taken from the \texttt{dining} case study from the benchmark 
suite of the tool.
Let us consider a program with two participants playing the role of a
dining philosopher who shares two forks with the other participant.

\begin{lstlisting}
philosopher(Fork1, Fork2) ->
  send req to Fork1,
  receive ack from Fork1,
  send req to Fork2,
  receive ack from Fork2,
  eat(),
  send release to Fork1,
  send release to Fork2,
  philosopher(Fork1, Fork2).

fork() ->
  receive req from Phil,
  send ack to Phil,
  receive release from Phil,
  fork().
\end{lstlisting}

The behavior of the philosophers is given by the pseudocode on the
right while pseudocode for the behavior of the forks is discussed below.
Each philosopher first acquires the forks (starting with the one on
its right, that is the one with the same index), then eats (with the
function call \lstinline{eat()} representing some terminating local
computation), and finally releases the forks before recurring.
Parameters \lstinline{Fork1} and \lstinline{Fork2} are references to
processes executing the behavior of forks described by the pseudocode on the left
that repeatedly waits for the request from process \lstinline{Phil},
\lstinline{ack}s the request, and waits for the \lstinline{release}
message from \lstinline{Phil}.

There are two possible behaviors of the system. The first (good) one where the philosophers alternate eating infinitely and a second (bad) behavior 
where both philosophers
manage to take only one fork each
resulting in a deadlock. 
Figure~\ref{graph:philosophers} depicts the global Choreography Automaton
representing the program above. 
We have two recursive executions where both philosophers
eat, represented as loops, and two executions which end in the same
deadlock state.
The deadlock is 
visible since State 5 is not final, and it has no outgoing transitions.
The automaton is simplified 
as it does not display the \texttt{ack} messages, and the
\texttt{release} messages are merged since their order is
irrelevant. The focus here is solely on the order of the
\texttt{req} messages. One can imagine to first extract the full
choreography automaton, and then merge equivalent behaviors and abstract 
away from uninteresting transitions.

\begin{figure}[t]
  \centering
  % \includegraphics[scale=.35]{images/deadlock.png}
  % \makebox[\textwidth][c]{
  \resizebox{0.9\textwidth}{!}{%
      \begin{tikzpicture}[node distance={40mm}, thick, main/.style = {draw, circle}] 
        \node[state] (n_1) {1};
        \node[state] (n_2) [below right of=n_1] {2};
        \node[state] (n_3) [below left of=n_1] {3};
        \node[state] (n_4) [right of=n_2] {4};
        \node[state] (n_5) [below of=n_1, fill=red] {5};
        \node[state] (n_6) [left of=n_3] {6};
        
        \draw[->] (n_1) -- node[midway, right, pos=0.5] {philo0→fork0:req} (n_2);
        \draw[->] (n_1) -- node[midway, left, pos=0.5] {philo1→fork1:req} (n_3);
        \draw[->] (n_2) -- node[midway, below right, pos=0.5] {philo1→fork1:req} (n_5);
        \draw[->] (n_3) -- node[midway, below left, pos=0.5] {philo0→fork0:req} (n_5);
        \draw[->] (n_2) -- node[midway, above, pos=0.5] {philo0→fork1:req} (n_4);
        \draw[->] (n_4) to[bend right=20] node[midway, right, pos=0.5] {philo0→fork\{0,1\}:release} (n_1);
        \draw[->] (n_3) -- node[midway, above, pos=0.5] {philo1→fork0:req} (n_6);
        \draw[->] (n_6) to[bend left=20] node[midway, left, pos=0.5] {philo1→fork\{0,1\}:release} (n_1);
      \end{tikzpicture}
    }
  % }%
  \caption{Global view of the dining philosophers example.}
  \label{graph:philosophers}
\end{figure}
This case study is taken from the \texttt{account} example from the benchmark 
suite of the tool.
We now consider a system where a bank account is accessed
by two clients, dubbed C1 and C2.
% 
\tikzset{
	hpath/.style={
			very thick,
			line cap = round,
			line join = round,
			line width=0.1cm,
			opacity=.70,
			color = teal!30
		}
}

\begin{lstlisting}
account(Value) ->
  receive
    read from Client ->
      send Value to Client,
      account(Value);
    NewValue from Client ->
      account(NewValue).

client() ->
  send read to Acc,
  receive Value from Acc,
  % operations on Value
  send NewValue to Acc.
\end{lstlisting}

The pseudocode yields the behavior of the bank account,
where $\mathsf{Value}$ represents its current balance.
This process waits for requests from a client.
A request can either be a \lstinline{read} access to know
the current balance or an update request of such value to a
\lstinline{NewValue}.

Symmetrically, client processes C1 and C2 behave according to the
pseudocode on the left: the process reads the current balance
from the account,
performs some internal operations based on such value, and
updates the balance.
%
The global view of the communicating system is depicted in
Figure~\ref{fig:account}.

\newcommand\dummy{C}
\begin{figure}[!ht]
	\centering
	\begin{tikzpicture}[node distance={27mm}, scale = .6, transform shape, thick, main/.style = {draw, circle}]
		\node (n_1) [state] {};
		\node (n_2) [state, below left of=n_1] {};
		\node (n_3) [state, below right of=n_1] {};
		\node (n_4) [state, left= 3.5cm of n_2] {4};
		\node (n_5) [state, below=3cm of n_1] {};
		\node (n_6) [state, right= 3.5cm of n_3] {6};
		\node (n_7) [state, below of=n_4] {7};
		\node (n_8) [state, below left of=n_5] {};
		\node (n_9) [state, below right of=n_5] {};
		\node (n_10) [state, below of=n_6] {10};
		\node (n_11) [state, accepting, below= of n_7] {11};
		\node (n_12) [state, accepting, below=1.5cm of n_8] {};
		\node (n_13) [state, accepting, below= 1.5cm of n_9] {};
		\node (n_14) [state, accepting, below=2.5cm of n_10] {14};

		\draw[->] (n_1) -- node[midway, above left] {acc→\dummy1:Value} (n_2);
		\draw[->] (n_2) -- node[midway, below left] {acc→\dummy2:Value} (n_5);
		\draw[->] (n_5) -- node[midway, above left=-2mm] {\dummy1→acc:NewValue} (n_8);
		\draw[->] (n_8) -- node[midway, below left] {\dummy2→acc:NewValue} (n_12);

		\draw[->] (n_1) -- node[midway, above right] {acc→\dummy2:Value} (n_3);
		\draw[->] (n_3) -- node[midway, below right] {acc→\dummy1:Value} (n_5);
		\draw[->] (n_5) -- node[midway, above right=-2mm] {\dummy2→acc:NewValue} (n_9);
		\draw[->] (n_9) -- node[midway, below right] {\dummy1→acc:NewValue} (n_13);

		\draw[hpath] ($(n_1.center) + (-5pt,-5pt)$) -- (n_2.center) -- (n_5.north west) -- ($(n_5.center)+(-5pt,0)$) -- (n_5.south west) -- (n_8.center) -- ($(n_12.center) + (0pt,5pt)$);
		\draw[hpath] ($(n_1.center) + (5pt,-5pt)$) -- (n_3.center) -- (n_5.north east)  -- ($(n_5.center)+(5pt,0)$) -- (n_5.south east)  -- (n_9.center) -- ($(n_13.center) + (0pt,5pt)$);

		\foreach \n in {1,2,3,5,8,9,12,13}{
				\node at (n_\n) {\n};
			}

		\draw[->] (n_2) -- node[midway, above=3mm] {C1→acc:NewValue} (n_4);
		\draw[->] (n_4) -- node[midway, above left] {acc→C2:Value} (n_7);
		\draw[->] (n_7) -- node[midway, above left] {C2→acc:NewValue} (n_11);

		\draw[->] (n_3) -- node[midway, above=3mm] {C2→acc:NewValue} (n_6);
		\draw[->] (n_6) -- node[midway, above right] {acc→C1:Value} (n_10);
		\draw[->] (n_10) -- node[midway, above right] {C1→acc:NewValue} (n_14);
	\end{tikzpicture}
	\caption{Global view of the bank account example}
	\label{fig:account}
\end{figure}

We can observe two correct executions where the operations are
performed in a read-update-read-update order (taking the path via
states 1-2-4-7-11 or the one via states 1-3-6-10-14).
%
However, there is also a read-read-update-update order on the
highlighted paths.
%
Although the choreography is not inherently incorrect, these
highlighted paths could represent a violation of mutual exclusion
which may be undesirable for developers in certain
contexts.
The choreography automaton in Figure~\ref{fig:account} helps in spotting
this issue.

%%% Local Variables:
%%% mode: LaTeX
%%% TeX-master: "main"
%%% End:


\subsubsection{Feature}
During this period, various features have been added to the tool:
\begin{itemize}  
    \item \textbf{Improved over-approximation in the global view}:  
    If, during computation, the value of some data is lost, the tool now  
    uses the \texttt{ANY} constant to indicate that it can represent any kind  
    of data. If a message is exchanged with \texttt{ANY} as its content, it  
    will match every possible receive branch, ensuring a more comprehensive  
    analysis.  
    
    \item \textbf{Basic value passing in the local view}:  
    The tool can now correctly parse programs where values are passed as  
    arguments to functions, allowing for a more accurate representation of  
    data flow between processes.  
    
    \item \textbf{Enhanced error handling}:  
    The tool is now more resilient to unexpected errors, reducing the  
    likelihood of crashes and ensuring that it provides meaningful output in  
    most cases, even when encountering invalid input.  
    
    \item \textbf{Improved CLI experience}:  
    A new library has been integrated for command-line argument management,  
    making the CLI more robust and user-friendly, with better parsing of  
    parameters and improved error messages.  

    \item \textbf{Improved testing environment and benchmark generation}:  
    The tool now produces useful benchmarking information, which is processed  
    by a dedicated Python script. This script collects and analyzes the data  
    to generate meaningful performance benchmarks. Additionally, if the  
    \texttt{correct\_gv.dot} file is present—containing the expected global  
    view—the script performs an automatic correctness check. This feature  
    lays the groundwork for more expressive and rigorous testing in the future,  
    enabling better validation of the tool's accuracy and performance across  
    different scenarios.  

\end{itemize}

\subsubsection{Bug Fix}
All of these improvements are made within the global view module:  
\begin{itemize}  
    \item \textbf{Fixed a bug causing false positive warnings}:  
    Resolved an issue where the tool incorrectly printed warnings about  
    failing to find the correct process during the emulation of a global view.  
    
    \item \textbf{Fixed a bug preventing full branch exploration}:  
    Addressed a problem that prevented the exploration of certain branches  
    during the global view emulation, ensuring a more comprehensive traversal  
    of all possible execution paths.  
    
    \item \textbf{Fixed a bug causing excessive duplicate branches}:  
    Corrected an issue where unnecessary duplicate branches were created  
    after receiving a message and defining a new variable. This fix improves  
    the accuracy of the global view by reducing redundant computations.  
\end{itemize}  


\subsubsection{Benchmarks}
Significant effort has been dedicated to developing a 
generic and automated script (written in Python) to test the tool and extract 
relevant information about the tool and its output.
Table~\ref{tab:gvbench} presents empirical data on the evaluation of various 
examples processed by the tool. 
The examples are primarily designed ad hoc to test specific aspects of the tool.
They are not sourced from well-known Erlang suites, as the tool is still in an
early stage and cannot yet parse most of them.
The columns of Table~\ref{tab:gvbench} provide the following information:

\begin{itemize}
    \item \textbf{Example}: Name of the test case.
    \item \textbf{Lines}: Number of lines of code in the example.
    \item \textbf{Tot LV}: Total number of local views generated by the example.
    \item \textbf{GV Nodes}: Number of nodes in the global view graph.
    \item \textbf{GV Edges}: Number of edges in the global view graph.
    \item \textbf{Warns}: Number of warnings raised.
    \item \textbf{Errors}: Number of errors detected (i.e. deadlocks).
    \item \textbf{Time}: Execution time in seconds.
\end{itemize}

The entries are ordered by the number of lines of code. The generated graphs 
included in the table are presented without minimization. Applying minimization 
techniques to the global view graph can lead to incorrect reductions in some 
cases, cause of a known bug. Therefore, we opted to maintain the full 
graph representation to ensure correctness.

\paragraph{Analysis of global view empirical data}
From the Table~\ref{tab:gvbench}, we can observe the overall behavior of the tool, 
focusing on the number of warnings and errors in the output. 
As previously mentioned, a high number of message exchanges directly correlates  
with an increased number of nodes and edges in the global view.
Warnings usually occur when the
tool encounters unknown elements in the source code, such as a new keyword or
an external function. They also appear when the tool loses track of a process
identifier, preventing it from mapping message exchanges correctly. 
This occurs in the \texttt{if-cases} example, which generates 185 warnings due to
the tool's failure to instantiate a variable with the correct process identifier,
making it unable to deliver the corresponding message. This issue likely stems 
from a bug in the actor emulation module.
Errors typically arise when an unexpected situation occurs or when a deadlock
state is detected. Most examples do not automatically terminate a process once
it completes its task (e.g., waiting for requests), which contributes to this
issue. However, one particularly notable case is \texttt{foo8}, which has 561
nodes, 560 edges, and 191 errors. The reason lies in its heavy use of case
constructs—some information is lost between the local and global views, leading
the tool to explore every possible execution path. This results in a high number
of states, edges, and undefined deadlock conditions. 
Using a minimized version of the local graph during actor emulation can
significantly reduce these values. However, this conflicts with the principle
of using an over-approximated approach. To address this, minimization could be
added as a program argument, allowing users to trade some generality for a
more precise global view graph.

\paragraph{About complexity}
A significant portion of the execution time is obviously consumed by the global  
view composition algorithm. This algorithm attempts to explore all possible  
program executions, considering every potential message order to meet the  
overapproximation requirement. In the worst-case scenario, without any  
restrictions, it must evaluate all permutations of messages, resulting in a  
factorial time complexity of \( O(n!) \). However, in most cases, more efficient  
strategies can be applied to mitigate this complexity.
The execution times show that our algorithm performs efficiently even on large 
examples (an example is considered large when there are many nodes and edges
because this corresponds to a high number of messages exchanged, leading to a
high number of possible combination of these messages), with the most complex 
cases (e.g., \texttt{foo8}) taking only a few seconds.

\begin{table}[!ht]
\centering
\begin{tabular}{|c|c|c|c|c|c|c|c|}
\hline
Example & Lines & Tot LV & GV Nodes & GV Edges & Warns & Errors & Time \\ 
\hline
unknown & 13 & 2 & 2 & 2 & 0 & 0 & 0.175s \\ 
foo9 & 14 & 4 & 4 & 3 & 1 & 3 & 0.187s \\ 
foo9c & 15 & 3 & 10 & 15 & 0 & 0 & 0.182s \\ 
pass & 16 & 3 & 3 & 2 & 0 & 0 & 0.174s \\ 
foo9d & 16 & 3 & 3 & 2 & 0 & 0 & 0.186s \\ 
funcall & 17 & 3 & 4 & 3 & 1 & 2 & 0.180s \\ 
forloop & 18 & 3 & 7 & 6 & 0 & 0 & 0.188s \\ 
foo1 & 18 & 3 & 8 & 7 & 0 & 0 & 0.175s \\ 
foo5 & 18 & 3 & 79 & 165 & 1 & 0 & 0.316s \\ 
async & 20 & 3 & 7 & 6 & 0 & 0 & 0.194s \\ 
foo4 & 20 & 4 & 16 & 19 & 0 & 2 & 0.182s \\ 
hof & 21 & 4 & 15 & 17 & 0 & 3 & 0.189s \\ 
foo9b & 21 & 4 & 4 & 4 & 14 & 1 & 0.183s \\ 
spawn & 22 & 3 & 9 & 8 & 0 & 0 & 0.179s \\ 
foo3 & 22 & 3 & 13 & 16 & 0 & 0 & 0.178s \\ 
account & 23 & 3 & 28 & 39 & 0 & 2 & 0.211s \\ 
airline & 23 & 3 & 15 & 26 & 1 & 0 & 0.232s \\ 
foo2 & 23 & 4 & 4 & 3 & 1 & 1 & 0.180s \\ 
foo9h & 23 & 4 & 24 & 35 & 0 & 5 & 0.196s \\ 
hello & 24 & 3 & 5 & 6 & 2 & 0 & 0.190s \\ 
trick & 24 & 4 & 9 & 9 & 0 & 0 & 0.184s \\ 
foo6 & 24 & 5 & 9 & 9 & 15 & 2 & 0.190s \\ 
foo9e & 24 & 5 & 14 & 14 & 0 & 5 & 0.186s \\ 
foo9f & 25 & 5 & 7 & 6 & 0 & 4 & 0.183s \\ 
foo9g & 25 & 5 & 44 & 83 & 0 & 7 & 0.226s \\ 
conditional & 26 & 2 & 25 & 24 & 1 & 16 & 0.197s \\ 
foo8 & 29 & 5 & 561 & 560 & 0 & 191 & 3.590s \\ 
producer & 30 & 4 & 11 & 10 & 0 & 1 & 0.183s \\ 
dining & 31 & 3 & 45 & 72 & 0 & 2 & 0.232s \\ 
ticktackloop & 32 & 4 & 6 & 6 & 2 & 0 & 0.182s \\ 
airline & 33 & 3 & 35 & 68 & 1 & 0 & 0.232s \\ 
ping & 36 & 3 & 6 & 5 & 1 & 0 & 0.188s \\ 
meViolation & 40 & 4 & 63 & 82 & 2 & 4 & 0.266s \\ 
serverclient & 41 & 5 & 9 & 8 & 8 & 3 & 0.187s \\ 
foo7 & 41 & 3 & 149 & 229 & 0 & 6 & 0.513s \\ 
ticktackstop & 46 & 5 & 19 & 27 & 7 & 0 & 0.214s \\ 
purchase & 47 & 5 & 49 & 66 & 6 & 0 & 0.257s \\ 
customer & 54 & 5 & 17 & 22 & 1 & 0 & 0.205s \\ 
if-cases & 57 & 4 & 148 & 210 & 185 & 30 & 0.525s \\ 
\hline
\end{tabular}
\caption{Global view empirical data}
\label{tab:gvbench}
\end{table}

\paragraph{Correctness of Global View}
One of the main contributions was enabling automatic verification of whether a
global view matches its correct version. Since knowing the complete specification
in advance is not trivial, we conducted tests on a few selected examples. 
The correct versions of the graph can be seen as what we expect to be the final
global view.
Table~\ref{tab:corrbench} briefly shows the correctness of the global view. 
The \textbf{Check} column indicates whether the global view 
correctly represents the expected behavior. Some examples, such as 
\texttt{async} and \texttt{ticktackloop}, are correctly generated because they 
are very simple example, while others fail. This may result from a missing 
feature in local or global view generation or from
the expected version not aligning with actual results. A key future goal of the
project is to generate as many correct global view versions as possible and to
systematically check and refine each one.

\begin{table}[!ht]
\centering
\begin{tabular}{|c|c|}
\hline
Example & Check \\ 
\hline
unknown & False \\ 
async & True \\ 
ticktackloop & True \\ 
ticktackstop & False \\ 
customer & False \\ 
\hline
\end{tabular}
\caption{Global view correctness data}
\label{tab:corrbench}
\end{table}

