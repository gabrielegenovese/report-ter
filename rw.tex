Software industry is
increasingly devoting attention to choreographic approaches
\cite{BPMN,bon18,fmmt20,DBLP:journals/software/AutiliIT15} because
they naturally support modularization and decoupling.
%
In fact, distributed components coordinate according to a global
description without the need of an explicit coordinator.
%
In this context, global specifications are crucial for guaranteeing
correctness (since they are blueprints of complex distributed systems
and feature model-driven development) as well as for program
comprehension.
%
To the best of our knowledge, the first attempts to distil global
specifications for message-passing systems goes back
to~\cite{myh09,lt12,lty15}.
%
These approaches aim to identify ``meaningful" global specifications
according to general properties such as deadlock-freedom or absence of
orphan messages~\cite{bz83}.
%
A limitation of these approaches is that they do not start from
components written in a full-fledged programming language.
%
Rather, distributed components are specified in~\cite{lt12,lty15} as
abstract models (respectively, $\pi$-calculus
processes~\cite{sw02,mil99,mpw92} and communicating finite-state
machines~\cite{bz83}).


%%% Local Variables:
%%% mode: LaTeX
%%% TeX-master: "main"
%%% End:
