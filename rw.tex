In the industrial context, Erlang's actor model has also inspired other 
programming languages or frameworks, such as Elixir, a 
programming language based on Erlang's virtual machine. Other programming 
languages also implement the actor model, such as Go with 
its GoRoutines (comparable to Erlang's spawns) and channels (similar to Erlang's
send and receive).
Software industry is increasingly devoting attention to choreographic approaches
\cite{BPMN,bon18,fmmt20,DBLP:journals/software/AutiliIT15} because
they naturally support modularization and decoupling.
%
In fact, distributed components coordinate according to a global
description without the need of an explicit coordinator.
%
In the academic context, research in the field of \textit{choreography} focuses 
on two main topics: \textit{choreography specification} and 
\textit{choreographic programming}.
\begin{itemize}
    \item \textit{Choreography specifications}: this area includes formal 
    methods, such as multiparty asynchronous session types 
    \cite{honda2008multiparty}, which have been established to describe the 
    interactive structure of a fixed number of actors from a global perspective.
    These methods enable the syntactic verification of actors' correctness by 
    projecting the global specification onto individual participants. 
    Choreography specifications are also studied as contracts, which provide 
    abstract descriptions of program behavior, known as \textit{multiparty 
    contracts} \cite{Huttel+16}.
    \item \textit{Choreographic programming}: this programming paradigm has been
    explored both theoretically, as in \cite{website:wscdl}, and industrially, 
    as in \cite{website:bpmn}. Several choreographic programming languages have
    been designed and studied to support this paradigm \cite{montesi2010jolie, 
    montesi2014choreographic, giallorenzo2020object, dalla2014aiocj}.
\end{itemize}
In this context, global specifications are crucial for guaranteeing
correctness (since they are blueprints of complex distributed systems
and feature model-driven development) as well as for program
comprehension. 
Most formal works on communication protocol 
specifications emphasize the projection of a global specification onto local 
specifications. However, the process of choreography extraction remains 
challenging and has been explored in \cite{cflm17}, with a general 
framework for extracting choreographies presented in 
\cite{cruz2022implementing}.
%
To the best of our knowledge, the first attempts to extract global
specifications for message-passing systems goes back
to~\cite{myh09,lt12,lty15}.
%
These approaches aim to identify ``meaningful" global specifications
according to general properties such as deadlock-freedom or absence of
orphan messages~\cite{bz83}.
%
A limitation of these approaches is that they do not start from
components written in a full-fledged programming language.
%
Rather, distributed components are specified in~\cite{lt12,lty15} as
abstract models (respectively, $\pi$-calculus
processes~\cite{sw02,mil99,mpw92} and communicating finite-state
machines~\cite{bz83}).


%%% Local Variables:
%%% mode: LaTeX
%%% TeX-master: "main"
%%% End:
