Explore and evaluate various models and paradigms that facilitate  
the development of robust and scalable concurrent applications is  
one of the primary aim of this project. We explore the bottom-up  
approaches that ``extract" global views from code. Note that  
bottom-up approaches exist~\cite{myh09,lt12,lty15,cflm17,cms18},  
but they have several limitations.  
%  
Firstly, they produce global views out of abstract models of local  
views (such as communicating-finite state machines~\cite{bz83} or  
some kind of behavioral types~\cite{Huttel+16}) and not from actual  
code written in a mainstream programming language (such as Erlang).  
Secondly, the extracted global views do faithfully capture the  
behavior of a local view only under some ``well-formedness"  
conditions. Crucially, this would hide buggy or unexpected behavior.  
%  
This is exactly the case where a global view would be most useful:  
the program is buggy and, in order to fix the bug, we need to  
understand the application-level protocol via a global abstract  
description.  
%  
For this project, we focus particularly on enhancing an existing  
tool that tries to extract choreographies with a bottom-up,  
over-approximated approach. In the next section, we'll define what  
are the requirements and the challenges to address this problem in  
the best way.