\label{sota}
\todo{già messo nel DoW, ci sta ripeterlo? io forse lo toglierei}
In the industrial context, Erlang's actor model has also inspired other 
programming languages or frameworks, such as Elixir \cite{website:elixir}, a 
programming language based on Erlang's virtual machine. Other programming 
languages also implement the actor model, such as Go \cite{website:golang} with 
its GoRoutines (comparable to Erlang's spawns) and channels (similar to Erlang's
send and receive).

\bigskip

In the academic context, research in the field of \textit{choreography} focuses 
on two main topics: \textit{choreography specification} and 
\textit{choreographic programming}.
\begin{itemize}
    \item \textit{Choreography specifications}: this area includes formal 
    methods, such as multiparty asynchronous session types 
    \cite{honda2008multiparty}, which have been established to describe the 
    interactive structure of a fixed number of actors from a global perspective.
    These methods enable the syntactic verification of actors' correctness by 
    projecting the global specification onto individual participants. 
    Choreography specifications are also studied as contracts, which provide 
    abstract descriptions of program behavior, known as \textit{multiparty 
    contracts} \cite{zava}.
    \item \textit{Choreographic programming}: this programming paradigm has been
    explored both theoretically, as in \cite{website:wscdl}, and industrially, 
    as in \cite{website:bpmn}. Several choreographic programming languages have
    been designed and studied to support this paradigm \cite{montesi2010jolie, 
    montesi2014choreographic, giallorenzo2020object, dalla2014aiocj}.
\end{itemize}

A way to formalize choreographies is through Choreographic Automata 
\cite{barbanerachoreography}, which describe a communication system using a 
finite-state automaton. By using this model, it is possible to use the results 
of automaton studies to demonstrate the mentioned properties 
\cite{orlando2021corinne}. Most formal works on communication protocol 
specifications emphasize the projection of a global specification onto local 
specifications. However, the process of choreography extraction remains 
challenging and has been explored in \cite{cruz2017paths}, with a general 
framework for extracting choreographies presented in 
\cite{cruz2022implementing}. In our work, we aim at extracting choreographies 
from an existing and widely-used programming language like Erlang, despite it 
not being originally designed with choreography in mind.