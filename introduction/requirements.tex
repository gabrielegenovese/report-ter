Below, we define the requirements that an automatic choreographic bottom-up approach should satisfy to enable its use for program understanding:
\begin{itemize}
    \item \textit{Bottom-up approach:} one should be able to automatically derive a choreography from code, so that it can be used  to help understand the code.
    \item \textit{Push-button technique:} the extraction of the choreography
    should be fully automatic, to be applicable to existing code
    without the need to add special annotations or any other input
    from the programmer.
    \item \textit{Always return a choreography:} even if the
         system is not well-behaved, hence its behavior can not be
         described by a choreography in the classical sense (since
         classical choreographies ensure by construction properties
         such as race and deadlock freedom). The extracted
         choreography should contain at least all the good behaviors,
         and possibly information on the not-well behaved parts.
    \item \textit{Highlight misbehavior:}
  debugging is our key reason to extract a choreographic description
  from code; therefore, extracted choreographies should explicitly flag
  misbehavior due to communications such as deadlocks, orphan
  messages, unspecified receptions, etc.
    \item \textit{Applicable to mainstream languages:} one should be able to
    extract the choreography from a real program written in a
    mainstream language. Natural targets are languages with a clean concurrency model and 		dedicated primitives for message-passing such as Erlang, Go, and Scala.
    \item \textit{Support creation and termination of participants:} in real
    message-passing systems new processes can be spawned, and some processes may
    terminate. Hence, a choreographic description should allow for
    a dynamic  number of participants.
    \item \textit{Support races:} races are disallowed by many choreographic
    approaches, yet are common in real programs. As such, they should
    be described (and possibly highlighted as potentially wrong in
    line with the idea of highlighting misbehavior), but not
    forbidden.
    \item \textit{Accessible yet precise notation:} choreographies
    should be represented with an intuitive, possibly
    graphical, formalism to improve readability. Instead of the usual algebraic
    formalisms one should appeal to graph-like notations such as
    labeled transition systems or finite state automata that pair a graphical representation with a
    well-defined  mathematical definition.
\end{itemize}