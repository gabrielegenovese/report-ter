The project began with the goal of creating a proof of concept, a prototype,  
for an analyzer that could extract choreography automata directly
from an existing codebase, which is quite unique in its field,
having a bottom-up over-approximation approach leaning to mainstream
programming languages.  
A significant amount of effort and resources have been invested,  
primarily in the tool's codebase and its design around examples and  
use cases. Additionally, extensive discussions have taken place  
regarding the high-level and general aspects of the tool.  

The results of this prototype are promising, demonstrating that such  
a tool is feasible and worth further exploration.  
We discussed on how the analysis should work, defining motivations,  
requirements, and challenges surrounding the tool. At the same time,  
we have introduced multiple improvements to the existing codebase to  
refine its output.  

Finally, efforts have been made to advance a formalization process,  
highlighting what has been done well and what still needs  
improvement. This has provided valuable insights into the tool’s  
design and future directions.  

The formalization process has just begun and will require further refinement  
in the future. A new testing method for the tool has been introduced, but  
additional effort is needed to generate correct global views. This is  
crucial for properly assessing the tool's precision and could lead to  
refinements in its existing components.  

The formalization process has also highlighted the need to refactor the  
local view component. A possible improvement is the addition of a parsing  
module capable of reading local views from files (e.g., DOT format).  
This enhancement would make the tool more general, removing its strict  
dependency on Erlang programs. 
With such a feature, developers could create a separate tool to translate  
their preferred programming language into a local view, outputting it  
as a DOT file. This file could then be provided as input to our tool  
to compute the corresponding global specification, making it a more  
versatile and widely applicable solution.  

Another area for improvement is the minimization and post-processing  
module. It can be expanded with better heuristics to generate clearer,  
more structured, and easily interpretable graphs. Enhancing these  
optimizations will improve readability and usability, making the tool’s  
output more insightful and practical.
